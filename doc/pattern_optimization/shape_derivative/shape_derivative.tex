\documentclass[10pt]{article}

\usepackage[latin1]{inputenc}
\usepackage{amsmath, amssymb, amsfonts, amsthm}
\usepackage{upgreek}
\usepackage{amsthm}
\usepackage{fullpage}
\usepackage{graphicx}
\usepackage{cancel}
\usepackage{subfigure}
\usepackage{mathrsfs}
\usepackage{outlines}
\usepackage[font={sf,it}, labelfont={sf,bf}, labelsep=space, belowskip=5pt]{caption}
\usepackage{hyperref}
% \usepackage{minted}
\usepackage{enumerate}
\usepackage{titling}

\usepackage{fancyhdr}
\usepackage[title]{appendix}

\DeclareMathOperator{\sgn}{sgn}

\pagestyle{fancy}
\headheight 24pt
\headsep    12pt
\lhead{Shape Derivatives for Functions of Homogenized Elasticity Tensors}
\rhead{\today}
\fancyfoot[C]{} % hide the default page number at the bottom
\lfoot{}
\rfoot{\thepage}
\renewcommand{\headrulewidth}{0.4pt}
\renewcommand\footrulewidth{0.4pt}
\providecommand{\abs}[1]{\lvert#1\rvert}
\providecommand{\norm}[1]{\lVert#1\rVert}
\providecommand{\dx}{\, \mathrm{d}x}
\providecommand{\dA}{\, \mathrm{d}A}
% \providecommand{\vint}[2]{\int_{#1} \! #2 \, \mathrm{d}x}
% \providecommand{\sint}[2]{\int_{\partial #1} \! #2 \, \mathrm{d}A}
\renewcommand{\div}{\nabla \cdot}
\providecommand{\shape}{\Omega(p)}
\providecommand{\mesh}{\mathcal{M}}
\providecommand{\boundary}{\partial \shape}
\providecommand{\vint}[1]{\int_{\shape} \! #1 \, \mathrm{d}x}
\providecommand{\sint}[1]{\int_{\boundary} \! #1 \, \mathrm{d}A}
\providecommand{\pder}[2]{\frac{\partial #1}{\partial #2}}
\providecommand{\tder}[2]{\frac{\mathrm{d} #1}{\mathrm{d} #2}}
\providecommand{\evalat}[2]{\left.#1\right|_{#2}}
\newcommand{\defeq}{\vcentcolon=}
\newtheorem{lemma}{Lemma}

\makeatletter
\usepackage{mathtools}
\newcases{mycases}{\quad}{%
  \hfil$\m@th\displaystyle{##}$}{$\m@th\displaystyle{##}$\hfil}{\lbrace}{.}
\makeatother

\setlength{\droptitle}{-80pt}
\title{Shape Derivatives for Functions of Homogenized Elasticity Tensors}
\author{Julian Panetta}

% BEGIN DOCUMENT
\begin{document}
\maketitle

We compute the shape derivative of a homogenized elasticity tensor and show how
it can be used to optimize various objective functions.

Recall from the periodic homogenization writeup that each component of
homogenized elasticity tensor for a microstructure $\omega$ in unit cell $Y$
can be written as an energy-like integral involving the fluctuation
displacements, ${\bf w}^{ij}$:
\begin{equation}
    \label{eqn:EhEnergy}
    C^H_{ijkl} = \frac{1}{|Y|} \int_\omega (e^{ij} + e({\bf w}^{ij})) : C : (e^{kl} + e({\bf w}^{kl})) \, \mathrm{d} {\bf y}.
\end{equation}
where the fluctuation displacements solve the cell problem:
\begin{equation}
\label{eqn:cell_constraint}
\begin{aligned}
     -\nabla \cdot (C : [e^{ij} + e({\bf w}^{ij})]) = {\bf 0} & \quad \text{in } \omega \\
{\bf \hat n} \cdot (C : [e^{ij} + e({\bf w}^{ij})]) = {\bf 0} & \quad \text{on } \partial \omega \setminus \partial Y \\
    {\bf w}^{ij}({\bf y})\ Y \text{-periodic} & \\
    \int_\omega \! {\bf w}^{ij}({\bf y})  \, \mathrm{d} {\bf y} =  {\bf 0}
\end{aligned}
\end{equation}
for each of the 6 (3 in 2D), constant strain basis tensors $e^{ij}$.

% Another way of phrasing this problem is to solve for full displacement ${\bf u}^{ij}$ such that:
% \begin{align*}
%      -\nabla \cdot (C : e({\bf u}^{ij})) = {\bf 0} & \quad \text{in } \omega \\
% {\bf \hat n} \cdot (C : e({\bf u}^{ij})) = {\bf 0} & \quad \text{on } \partial \omega \setminus \partial Y \\
%     \int_\omega \! {\bf u}^{ij}({\bf y})  \, \mathrm{d} {\bf y} =  {\bf 0}, \quad
%     \frac{1}{|Y|} \int_\omega \! e({\bf u}^{ij})  & \, \mathrm{d} {\bf y} =  e^{ij} \\
%     e({\bf u}^{ij}) \quad Y \text{-periodic} &
% \end{align*}
% This solution is called a ``linear plus periodic'' displacement because  up to
% a rigid translation, ${\bf u}^{ij}({\bf y}) = {\bf w}^{ij} + e^{ij} {\bf y}$.
% This formulation is nice because first it allows us to express the homogenized
% coefficients as even more energy-like terms:
% \begin{equation}
%     \label{eqn:EhEnergy}
%     C^H_{ijkl} = \frac{1}{|Y|} \int_Y e({\bf u}^{ij}) : C : e({\bf u}^{kl}) \, \mathrm{d} {\bf y}.
% \end{equation}
% Second, the new cell problem is a 

\section{Shape Derivative of $C^H_{ijkl}$}
Because energy functionals are self-adjoint, the shape derivative of
(\ref{eqn:EhEnergy}) turns out to be surprisingly simple. We start by computing
the change in $C^H_{ijkl}$ caused by an arbitrary boundary perturbation, then
give the result for parametrized microstructure boundaries.

\subsection{Arbitrary Boundary Perturbations}
Consider a perturbation of the shape's boundary, $\delta\omega$, caused by
advecting the boundary with an infinitesimal velocity field ${\bf v}$. The
resulting variation of $C^H_{ijkl}$ for $ij \ne kl$ ($ij = kl$ gets an
identical final result by product rule) is:
$$
\delta C^H_{ijkl} = \left<\pder{C^H_{ijkl}}{\omega}, \delta \omega\right>
    + \left<\pder{C^H_{ijkl}}{{\bf w}^{ij}}, \delta {\bf w}^{ij}\right>
    + \left<\pder{C^H_{ijkl}}{{\bf w}^{kl}}, \delta {\bf w}^{kl}\right>.
$$
Consider the linear functional
$\left<\pder{C^H_{ijkl}}{{\bf w}^{ij}}, \cdot \right>$  on an arbitrary
admissible perturbation of ${\bf w}^{ij}$ (i.e. periodic and with no rigid
translation component), $\phi$:
$$
\left<\pder{C^H_{ijkl}}{{\bf w}^{ij}}, \phi\right> =
\lim_{h \to 0} \, \tder{}{h} \frac{1}{|Y|} \int_\omega (e^{ij} + e({\bf w}^{ij} + h \phi)) : C : (e^{kl} + e({\bf w}^{kl})) \, \mathrm{d} {\bf y}
$$
Differentiating under the integral and using the linearity of strain, this is:
$$
\left<\pder{C^H_{ijkl}}{{\bf w}^{ij}}, \phi\right> =
\frac{1}{|Y|} \int_\omega e(\phi) : C : (e^{kl} + e({\bf w}^{kl})) \, \mathrm{d} {\bf y}
$$
Then, integrating by parts to move the strain off $\phi$:
$$
\left<\pder{C^H_{ijkl}}{{\bf w}^{ij}}, \phi\right> =
-\frac{1}{|Y|} \int_\omega \phi \cdot \left(\nabla \cdot \left[C : (e^{kl} + e({\bf w}^{kl}))\right]\right) \, \mathrm{d} {\bf y}
+ \frac{1}{|Y|} \int_{\partial \omega} \phi \cdot \left(\hat{{\bf n}} \cdot \left[C : (e^{kl} + e({\bf w}^{kl}))\right]\right) \, \mathrm{d} {\bf y}
$$
The volume integral vanishes because ${\bf w}^{kl}$ solves the $kl^\text{th}$
cell problem. The $\partial \omega \setminus \partial Y$ portion of the surface
integral vanishes for the same reason, and the $\partial \omega \cup \partial
Y$ portion vanishes because $\phi$, $C$, $e^{kl}$, and $e({\bf w}^{kl})$ are all
either constant or periodic. The same argument holds for
$\left<\pder{C^H_{ijkl}}{{\bf w}^{kl}}, \phi\right>$, so we have
$$
\left<\pder{C^H_{ijkl}}{{\bf w}^{ij}}, \phi\right> =
\left<\pder{C^H_{ijkl}}{{\bf w}^{kl}}, \phi\right> = 0
$$
without solving any adjoint problem!
Thus Reynold's transport theorem gives the full shape derivative:
$$
\dot{C}^H_{ijkl}[{\bf v}] \defeq
\delta C^H_{ijkl} = \left<\pder{C^H_{ijkl}}{\omega}, \delta \omega\right> =
\boxed{
\frac{1}{|Y|} \int_{\partial \omega} \left[(e^{ij} + e({\bf w}^{ij})) : C : (e^{kl} + e({\bf w}^{kl}))\right] {\bf v} \cdot {\bf \hat{n}} \dA({\bf y}).}
$$

\subsection{Parametrized Boundary Perturbations}
In our setting, the microstructure boundary $\partial \omega$ is parametrized by a vector ${\bf p}$,
consisting of e.g. wire mesh node offsets and thicknesses. This parametrization is smooth
in the sense that we can define parameter velocity fields ${\bf v}_{p_i}({\bf y}, {\bf
p})$ at almost all points ${\bf y} \in \partial \omega({\bf p}) \setminus \partial Y$.
These velocity fields give how the point ${\bf y}$ moves to stay on the
boundary as $p_a$ changes. The derivative of $C^H_{ijkl}$ with respect to
parameter $p_a$ is just $\dot{C}^H_{ijkl}$ evaluated on the ${\bf v}_{p_a}$
velocity field:
\begin{equation}
\label{eqn:paramDerivative}
\pder{C^H_{ijkl}}{p_a} =
\frac{1}{|Y|} \int_{\partial \omega} \left[(e^{ij} + e({\bf w}^{ij})) : C : (e^{kl} + e({\bf w}^{kl}))\right] {\bf v}_{p_a} \cdot {\bf \hat{n}} \dA({\bf y}).
\end{equation}
Thus, we can compute the derivative of each homogenized coefficient with
respect to each parameter $p_a$ provided that we know the normal velocity
scalar field induced by changing $p_a$, ${\bf \hat{n}} \cdot {\bf v}_{p_a}$.

\subsection{Discretization}
If linear finite elements are used, (\ref{eqn:paramDerivative}) is
particularly easy to compute; the energy density term
$(e^{ij} + e({\bf w}^{ij})) : C : (e^{kl} + e({\bf w}^{kl}))$
is constant on each boundary element and can be stored as a per-boundary-element tensor $C^e_{ijkl}$.
Then the integral can be computed as:
$$
\pder{C^H_{ijkl}}{p_a} =
\sum_e \frac{1}{|Y|} C^e_{ijkl} \int_e {\bf v}_{p_a} \cdot \hat{{\bf n}} \dA({\bf y}) := \frac{1}{|Y|} C^e_{ijkl} v^e,
$$
where $v^e$ is the integral of parameter $p_a$'s (linear) normal velocity field
over boundary element $e$. In other words, computing how a particular
$C_{ijkl}$ changes with respect to $p_a$ is just a dot product.

If higher order elements are used, we can still use this trick if we
approximate either the energy density term or the normal velocity field as
piecewise constant. However, to compute the integral exactly requires
quadrature over each boundary element.

\section{Objective Functions}
\subsection{Elasticity Tensor Fit}
The simplest objective function is the squared Frobenius norm distance to a
target tensor $C^*$:
$$
J_C = \frac{1}{2} ||C^H - C^*||^2_F = \frac{1}{2} \left[C^H - C^*\right]_{ijkl} \left[C^H - C^*\right]_{ijkl}
$$
Then
$$
\pder{J_C}{p_a} = \left[C^H - C^*\right]_{ijkl} \pder{C^H_{ijkl}}{p_a}.
$$

Unfortunately, as discussed in the ``Objectives for Pattern Parameter
Optimization'' writeup, this objective is poorly behaved. A better one uses
the compliance tensor, $S = C^{-1}$, or some function of it's entries (e.g. Young's
modulus, Poisson ratio).

\subsection{Compliance Tensor Fit}
Inverting rank $4$ tensors requires a bit more than just inverting the
flattened tensor representation; see the ``Tensor Flattening'' writeup for the
details. But following the formula there, we can differentiate the inverted
tensor $S = C^{-1}$:
\begin{align*}
    F(C : S) &= F(I) \quad \Longrightarrow \\
    F(C) \mathscr{D} F(S) &= \mathscr{D}^{-1} \quad \Longrightarrow \\
    \pder{}{p_a} \left[ F(C) \mathscr{D} F(S) \right] &= \pder{F(C)}{p_a} \mathscr{D} F(S) + F(C) \mathscr{D} \pder{F(S)}{p_a} = 0 \Longrightarrow \\
    \pder{F(S)}{p_a} = -\mathscr{D}^{-1} F(C)^{-1} \pder{F(C)}{p_a} \mathscr{D} F(S) &= -F(S)\mathscr{D} \pder{F(C)}{p_a} \mathscr{D} F(S) = -F\left(S : \pder{F(C)}{p_a} : S\right),
\end{align*}
where we used $F(C)^{-1} = \mathscr{D} F(S) \mathscr{D}$, which follows
directly from the inversion formula in the tensor flattening writeup. The
derivative of the compliance tensor, $\pder{S^H}{p_a}$, can then be read off by
unflattening $\pder{F(S)}{p_a}$.

Now the squared Frobenius norm distance to a target compliance tensor $S^*$ can
be differentiated:
\begin{align*}
    J_S &= \frac{1}{2} ||S^H - S^*||^2_F \\
    \pder{J_S}{p_a} &= \left[S^H - S^*\right]_{ijkl} \pder{S^H_{ijkl}}{p_a}.
\end{align*}
As mentioned in the objectives writeup, it may instead be desirable to use some
other weighting of the Frobenius norm terms. For example, we could use Frobenius
norm distance in flattened form:
\begin{align*}
    J_{F(S)} &= \frac{1}{2} \norm{F(S) - F(S^*)}^2_F =
        \frac{1}{2} \left[F(S) - F(S^*)\right]_{ij}
                    \left[F(S) - F(S^*)\right]_{ij} \\
                    \pder{J_{D^{-1}}}{p_a} &= [D^{-1} - {D^*}^{-1}]_{ij}
                    \left[\pder{D^{-1}}{p_a}\right]_{ij} = 
        \pder{J_{F(S)}}{p_a} = \left[F(S) - F(S^*)\right]_{ij}
                    \left[\pder{F(S)}{p_a}\right]_{ij}.
\end{align*}

\subsection{Elastic Moduli Fit}
For objectives that can be written as functions of the Elastic moduli (or for
finding extremal moduli), it's also useful to differentiate the moduli.
To simplify notation for the following, we are going to overload $S$ to be the
flattened homogenized compliance tensor.

Recall that for orthotropic, axis-aligned materials, the flattened compliance
tensor looks like:
$$
F(S) = \begin{pmatrix}
    \frac{1}{E_x} & -\frac{\nu_{yx}}{E_y} & -\frac{\nu_{zx}}{E_z} & 0 & 0 & 0 \\
    -\frac{\nu_{yx}}{E_y} & \frac{1}{E_y} & -\frac{\nu_{zy}}{E_z} & 0 & 0 & 0 \\
    -\frac{\nu_{zx}}{E_z} & -\frac{\nu_{zy}}{E_z} & \frac{1}{E_z} & 0 & 0 & 0 \\
    0 & 0 & 0 & \frac{1}{4 \mu_{yz}} & 0 & 0 \\
    0 & 0 & 0 & 0 & \frac{1}{4 \mu_{zx}} & 0 \\
    0 & 0 & 0 & 0 & 0 & \frac{1}{4 \mu_{xy}}
\end{pmatrix}.
$$
The orthotropic moduli can therefore be written as simple functions of the
compliance tensor entries. For clarity, we overload $S$ and $C$ to mean their
flattened counterparts when indexed with only 2 indices (i.e. $S_{ij} \defeq
F(S)_{ij}$, $C_{ij} \defeq F(C)_{ij}$):
\begin{gather*}
E_x = \frac{1}{S_{00}}, \quad
E_y = \frac{1}{S_{11}}, \quad
E_z = \frac{1}{S_{22}}, \\
\nu_{yx} = -E_y S_{01}, \quad
\nu_{zx} = -E_z S_{02}, \quad
\nu_{zy} = -E_z S_{12}, \\
\mu_{yz} = \frac{1}{4 S_{33}}, \quad
\mu_{zx} = \frac{1}{4 S_{44}}, \quad
\mu_{xy} = \frac{1}{4 S_{55}}.
\end{gather*}
Thus:
\begin{gather*}
    \pder{E_x}{p_a} = -\frac{1}{S_{00}^2} \pder{S_{00}}{p_a}, \quad
    \pder{E_y}{p_a} = -\frac{1}{S_{11}^2} \pder{S_{11}}{p_a}, \quad
    \pder{E_z}{p_a} = -\frac{1}{S_{22}^2} \pder{S_{22}}{p_a}, \\
    \pder{\nu_{yx}}{p_a} = \frac{S_{01}}{S_{11}^2} \pder{S_{11}}{p_a} - E_y \pder{S_{01}}{p_a}, \quad
    \pder{\nu_{zx}}{p_a} = \frac{S_{02}}{S_{22}^2} \pder{S_{22}}{p_a} - E_y \pder{S_{02}}{p_a}, \quad
    \pder{\nu_{zy}}{p_a} = \frac{S_{12}}{S_{22}^2} \pder{S_{22}}{p_a} - E_y \pder{S_{12}}{p_a}, \\
    \pder{\mu_{yx}}{p_a} = -\frac{1}{4 S_{33}^2} \pder{S_{33}}{p_a}, \quad
    \pder{\mu_{zx}}{p_a} = -\frac{1}{4 S_{44}^2} \pder{S_{44}}{p_a}, \quad
    \pder{\mu_{xy}}{p_a} = -\frac{1}{4 S_{55}^2} \pder{S_{55}}{p_a}.
\end{gather*}
Since shear moduli also appear on the lower diagonal of $F(C)$, we can express
their shape derivatives more simply as:
$$
    \pder{\mu_{yx}}{p_a} = \pder{C_{33}}{p_a}, \quad
    \pder{\mu_{zx}}{p_a} = \pder{C_{44}}{p_a}, \quad
    \pder{\mu_{xy}}{p_a} = \pder{C_{55}}{p_a}.
$$

\end{document}
