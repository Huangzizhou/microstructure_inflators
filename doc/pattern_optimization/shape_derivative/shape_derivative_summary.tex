\documentclass[twocolumn,10pt]{article}

\usepackage[utf8]{inputenc}
\usepackage{amsmath, amssymb, amsfonts, amsthm}
\usepackage{upgreek}
\usepackage{amsthm}
\usepackage{fullpage}
\usepackage{graphicx}
\usepackage{cancel}
\usepackage{subfigure}
\usepackage{mathrsfs}
\usepackage{enumerate}
%\usepackage{outlines}
\usepackage[font={sf,it}, labelfont={sf,bf}, labelsep=space, belowskip=5pt]{caption}
\usepackage{hyperref}
% \usepackage{minted}

\usepackage{float}
% \floatplacement{figure}{H}

\usepackage{fancyhdr}
\usepackage[title]{appendix}
\usepackage{siunitx}

\DeclareMathOperator{\tr}{tr}
\DeclareMathOperator{\sgn}{sgn}
\DeclareMathOperator{\sinc}{sinc}
\DeclareMathOperator{\rref}{rref}
\DeclareMathOperator{\cof}{cof}

\providecommand{\abs}[1]{\lvert#1\rvert}
\providecommand{\norm}[1]{\lVert#1\rVert}
\providecommand{\dx}{\, \mathrm{d}x}
\providecommand{\dA}{\, \mathrm{d}A}
% \providecommand{\vint}[2]{\int_{#1} \! #2 \, \mathrm{d}x}
% \providecommand{\sint}[2]{\int_{\partial #1} \! #2 \, \mathrm{d}A}
\renewcommand{\div}{\nabla \cdot}
\providecommand{\e}{\epsilon}
\providecommand{\shape}{\Omega(p)}
\providecommand{\boundary}{\partial \shape}
\providecommand{\vint}[1]{\int_{\shape} \! #1 \, \mathrm{d}x}
\providecommand{\sint}[1]{\int_{\boundary} \! #1 \, \mathrm{d}A}
\providecommand{\pder}[2]{\frac{\partial #1}{\partial #2}}
\providecommand{\tder}[2]{\frac{\mathrm{d} #1}{\mathrm{d} #2}}
\providecommand{\evalat}[2]{\left.#1\right|_{#2}}
\newcommand{\defeq}{\vcentcolon=}
\newtheorem{lemma}{Lemma}
\newcommand\numberthis{\addtocounter{equation}{1}\tag{\theequation}}

\makeatletter
\usepackage{mathtools}
\newcases{mycases}{\quad}{%
  \hfil$\m@th\displaystyle{##}$}{$\m@th\displaystyle{##}$\hfil}{\lbrace}{.}
\makeatother
\DeclarePairedDelimiter\ceil{\lceil}{\rceil}
\DeclarePairedDelimiter\floor{\lfloor}{\rfloor}

\newenvironment{amatrix}[1]{%
  \left[\begin{array}{@{}*{#1}{c}|c@{}}
}{%
  \end{array}\right]
}

%% + Abtin
\usepackage{fullpage}
\usepackage[usenames,dvipsnames]{color}
\usepackage{paralist}
\usepackage{prettyref}
\newrefformat{sec}{Section~\ref{#1}}
\newrefformat{tbl}{Table~\ref{#1}}
\newrefformat{fig}{Fig.~\ref{#1}}
\newrefformat{chp}{Chapter~\ref{#1}}
\newrefformat{eqn}{Eq.~\eqref{#1}}
\newrefformat{set}{Eq.~Set~\eqref{#1}}
\newrefformat{alg}{Algorithm~\ref{#1}}
\newrefformat{apx}{Appendix~\ref{#1}}
\newcommand\pr[1]{\prettyref{#1}}

\newcommand\todo[1]{\textcolor{magenta}{\bf [TODO: #1]}}
\renewcommand\vec[1]{\ensuremath{\mathbf #1}}
\def\x{\vec{x}}
\def\y{\vec{y}}
\def\u{\vec{u}}
\def\ue{\vec{u}^\e}
\def\strain{\varepsilon}


\begin{document}
The minimization problem is generically defined as \note{place-holder;
  needs constraints etc.}
\begin{equation}
  \label{eqn:min}
  \argmin_\text{admissible $\omega$} J(\omega),
\end{equation}
where $J$ is some deviation metric on the micro structure. Letting $S$
denote the compliance tensor, we choose
\begin{equation}
  \label{eqn:compObjective}
  J(\omega) = \frac{1}{2}\norm{S^H(\omega) - S^*}^2_F,
\end{equation}
for a microstructure with shape $\omega$. There are several other
possible choices for the objective functional $J$, such as deviation
of elasticity tensor or error in displacement. In our setting, we are
interested in a particular deformation response and compliance becomes
the natural measure in this context \note{justify why not displacement
  itself}. A nice side effect of defining $J$ as in
\pr{eqn:compObjective} is that $J$ is self-adjoint and as a result the
shape derivative will have a simple form.

The microstructure boundary $\partial \omega$ is parameterized by a
vector $\vec{p}$, consisting of, for instance, wire mesh node offsets
and thicknesses. The variation of $\partial \omega$ with respect to
$\vec{p}$
\begin{equation}
  \vec{v}_{p_\alpha}(\y,\vec{p}) = \pder{\y}{p_\alpha}\quad\text{for } \y\in \partial\omega.
\end{equation}
(with proper assumptions) defines perturbation velocity fields over
the boundary. Using $\vec{p}$ the minimization problem can be written
as
\begin{equation}
  \label{eqn:minParam}
  \argmin_\text{admissible $\vec{p}$} J(\vec{p})\text{ where }
  J(\vec{p}) = \frac{1}{2}\norm{S^H(\vec{p}) - S^*}^2_F.
\end{equation}

The derivative of the objective function is then
\begin{equation}
  \pder{J}{p_\alpha} = \left[S^H - S^*\right]_{ijkl} \pder{S^H_{ijkl}}{p_\alpha}.
\end{equation}

%%%%%%%
The derivative of $C^H_{ijkl}$ with respect to parameter $p_a$ is just
$\d C^H_{ijkl}$ evaluated on the $\vec{v}_{p_a}$ velocity field. Thus,
we can compute the derivative of each homogenized coefficient with
respect to each parameter $p_a$ provided that we know the normal
velocity scalar field induced by changing $p_a$, $\vec{v}_{p_a} \cdot
\hat{\vec{n}}$.
%%%%%%%

In finding a solution to \pr{eqn:minParam}, the shape derivative of a
homogenized elasticity tensor is required. The shape derivative of the
elasticity tensor for microstructure with shape $\omega$ and
perturbation $\vec{v}$ is defined as the G\^ateaux derivative
\cite{zolesio2001shapes}
\begin{equation}
  \d C^H[\vec{v}] \defeq \lim_{t\downarrow 0} \frac{C^H((1+t\vec{v})\omega)-C^H(\omega)}{t},
\end{equation}
where $(1+t\vec{v})\omega \defeq \{\x+t\vec{v} : \x \in \omega\}$.

The homogenized elasticity tensor \note{give ref}, with a bit of
manipulation, can be rewritten in the energy form as
\begin{equation}
  \label{eqn:EhEnergy}
  C^H_{ijkl} = \frac{1}{|Y|} \int_\omega (\vec{e}^{ij} +
  \strain(\vec{w}^{ij})) : C^\text{base} : (\vec{e}^{kl} +
  \strain(\vec{w}^{kl})) \, \mathrm{d} \y.
\end{equation}
Using this form, the shape derivative of $C^H$ can be readily written as
\begin{align}
  \label{eqn:paramDerivative}
  \d C^H_{ijkl}[\vec{v}] =
  \frac{1}{|Y|} \int_{\partial \omega} &[(\vec{e}^{ij} + \strain(\vec{w}^{ij})) : C^\text{base}\\
    &: (\vec{e}^{kl} + \strain(\vec{w}^{kl}))] (\vec{v} \cdot \hat{\vec{n}}) \dA(\y).\notag
\end{align}

\para{Discretization} If linear finite elements are used,
(\ref{eqn:paramDerivative}) is particularly easy to compute. The
energy density term is constant on each boundary element and can be
stored as a per-boundary-element tensor $C^e_{ijkl}$.  Then the
integral can be computed as:
\begin{equation}
\pder{C^H_{ijkl}}{p_\alpha} =
\sum_\alpha \frac{1}{|Y|} C^e_{ijkl} \int_e \vec{v}_{p_\alpha} \cdot \hat{{\bf n}} \dA({\y}). % = \frac{1}{|Y|} C^e_{ijkl} v^e,
\end{equation}
%% where $v^e$ is the integral of parameter $p_a$'s (linear) normal velocity field
%% over boundary element $e$. In other words, computing how a particular
%% $C_{ijkl}$ changes with respect to $p_a$ is just a dot product.
If higher order elements are used, we can still approximate either the
energy density term or the normal velocity field as piecewise
constant. However, to compute the integral exactly requires quadrature
over each boundary element.

\para{Computing the compliance tensor}We perform the computation on
the flattened tensor representation. Letting $F$ denote the flattening
operator \note{needs def}, the derivative of the compliance tensor
$S=C^{-1}$ is given by
\begin{align}
    \pder{S}{p_\alpha} = -S : \pder{F(C)}{p_\alpha} : S.
\end{align}
\note{check correctness with Julian}

\bibliographystyle{plain}
\bibliography{References}

\end{document}
