\section{Implementation}
This section summarizes the all sections above in terms of work needed.  This
aims to be a giant to-do list.

There are essentially two independent modules:
\begin{description}
\item{\bf Convert arbitrary elasticity tensors into pattern parameters}\\
{\it Input:} An array of elasticity tensors.

{\it Output:} An array of pattern parameters (and fitting errors). 

{\it Validation:} If the homogenized elasticity tensor has a closed formula, we
should validate it against a fine-scale simulation. We also should use this
stage to generate a bound for all possible material properties that can be
generated with a given type of pattern.

{\it Physical experiments:}  Validate homogenization formula on simple patterns.\\

Approach 1: Look-up Table
\begin{itemize}
\item For microstructures with closed formulas for homogenized elasticity
    tensors, sample the space of all possible pattern parameter combinations.
\item For microstructures without closed formulas, use the representative volume
method (numerical coarsening) to find them.  I.e., for all possible pattern
parameters, generate a cube of microstructures and simulate with periodic
boundary conditions its behavior using a very fine mesh or mesh-free FEM.
Use \cite{Kharevych2009} to obtain the homogenized elasticity tensors.
\item Build a table that supports fast lookup of pattern parameters by
    elasticity tensor. Return an interpolation of closest entries using a
    reasonable distance measure.
\item Optionally run a few iterations of the optimization described in Approach
      2 to improve the fit.
\end{itemize}

Approach 2: Direct Optimization of Material Parameters.
\begin{itemize}
\item For microstructures with closed formulas for homogenized elasticity
    tensors, directly optimize the Frobenius distance from the target elasticity
    tensor with a nonlinear optimization algorithm, e.g. Newton's method.
\end{itemize}

\item{\bf Generate microstructure geometry}\\
{\it Input:} An array of pattern parameters.

{\it Output:} Microstructure mesh.

{\it Validation:} Simulate the synthesized microstructure to see if behaves like
a homogenization of the pattern.

{\it Physical experiments:} Validate generated patterns in experiments.

Approach 1: Tile pattern for each element and stitch.
\begin{itemize}
\item Write code to generate regular tiled patch for each element.
\item Write code to stitch patches together without significantly effect the material
properties.
\item Verify stitching does not introduce artifacts globally.
\end{itemize}

Approach 2: Tracing dual graph and place holes.
\begin{itemize}
\item Implement or borrow code for streamline tracing.  Make sure streamlines
form a network.
\item Write code to procedurally generate holes at each stream line
intersection.
\item Verify the resulting mesh is a valid microstructure.
\end{itemize}

Approach 3: Isosurface synthesis.
\begin{itemize}
    \item Register the per-element lamination directions.
    \item Find $p$ scalar fields whose isosurface normals fit to the lamination
        directions.
    \item Generate a single mesh of the unioned isosurface extrusions.
\end{itemize}
\end{description}

There is also work involved in generating results and validations.
\begin{itemize}
\item Extend the mesh-free FEM simulation to 3D.
\item Material property painting program for generating inputs.
\item Code for computing homogenized elasticity tensors from
    \cite{Kharevych2009}. We need this to generate an elasticity tensor given a
    load specification and a deformation field.
\end{itemize}



