\section{Implementation}
This section summarizes the all sections above in terms of work needed.  This
would be a giant to-do list.

There are mainly two independent modules:
\begin{description}
\item{\bf Convert arbitrary material tensor into pattern parameters}\\
{\it Input:} An array of material tensors.

{\it Output:} An array of pattern parameters (and fitting errors).  A bound for
all possible material properties that can be generated using a given type of patterns.

{\it Validation:} If homogenization has closed formula, we should validate it
with simulation.

{\it Physical experiments:}  Validate homogenization formula on simple patterns.\\

Approach 1: look up table.
\begin{itemize}
\item For microstructure with closed formula of homogenized material properties,
sample the space of all possible pattern parameter combinations.
\item For microstructure without closed formula of homogenized material
properties, use representative volume method to find it out.  I.e. For all
possible pattern parameters, generate a
cube made of microstructures, simulate it either on a very fine mesh, or use
Mesh-free techniques, or use \cite{Kharevych2009} to obtain the homogenized
material property.
\item Compute the homogenized material properties for each pattern parameter.
\item Build a table that support fast lookup with material tensor as key.  If
the tensor is not in the table, return the closest entry using a reasonable
measure.
\end{itemize}

Approach 2: optimization of material parameters.
\begin{itemize}
\item For microstructure with closed formula, using optimization to find the
best fit pattern parameter.
\end{itemize}

\item{\bf Generate microstructure geometry}\\
{\it Input:} An array of pattern parameters.

{\it Output:} Microstructure mesh

{\it Validation:} Simulate the resulting microstructure to see if its behaviors
are expected.

{\it Physical experiments:} Validate generated patterns in experiments.

Approach 1: tile pattern for each element and stitch.
\begin{itemize}
\item Write code to generate regular tiled patch for each element.
\item Write code to stitch patches together without significantly effect the material
properties.
\item Verify stitching does not introduce artifacts globally.
\end{itemize}

Approach 2: tracing dual graph and pack holes.
\begin{itemize}
\item Implement or borrow code for streamline tracing.  Make sure streamlines
form a network.
\item Write code to procedurally generate holes at each stream line
intersection.
\item Verify the resulting mesh is a valid microstructure.
\end{itemize}
\end{description}



There is also work involved in generating results and validations.
\begin{itemize}
\item Mesh-free 3D simulation.
\item Material property painting program.  It would be handy to generate
inputs.
\item Code for computing homogenized material property from
\cite{Kharevych2009}.  This would be helpful to generate material tensor from
load specification and deformation field.
\end{itemize}



