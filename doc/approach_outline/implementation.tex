\section{Implementation}
This section summarizes the all sections above in terms of work.

\subsection{Software}
The software would consist two independent modules:
\begin{description}
\item{\bf Convert arbitrary material tensor into pattern parameters}\\
{\it Input:} An array of material tensors.

{\it Output:} An array of pattern parameters (and fitting errors).\\

Approach 1: look up table.
\begin{itemize}
\item For microstructure with closed formula of homogenized material properties,
sample the space of all possible pattern parameter combinations.
\item For microstructure without closed formula of homogenized material
properties, use representative volume method to find it out.  I.e. For all
possible pattern parameters, generate a
cube made of microstructures, simulate it either on a very fine mesh, or use
Mesh-free techniques, or use \cite{Kharevych2009} to obtain the homogenized
material property.
\item For each sampled pattern parameter, compute the homogenized material
tensor.
\item Build a table that support fast lookup with material tensor as key.  If
the tensor is not in the table, return the closest entry using a reasonable
measure.
\end{itemize}

Approach 2: optimization of material parameters.



\item{\bf Generate microstructure geometry}\\
{\it Input:} An array of pattern parameters.

{\it Output:} Microstructure mesh


\end{description}

\subsection{Experiments}
Due to the slow turn around time, we have to plan ahead.
