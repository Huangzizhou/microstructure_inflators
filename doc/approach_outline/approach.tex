\documentclass[10pt]{article}

\usepackage[latin1]{inputenc}
\usepackage{amsmath, amssymb, amsfonts, amsthm}
\usepackage{upgreek}
\usepackage{amsthm}
\usepackage{fullpage}
\usepackage{graphicx}
\usepackage{cancel}
\usepackage{subfigure}
\usepackage{mathrsfs}
\usepackage{outlines}
\usepackage[font={sf,it}, labelfont={sf,bf}, labelsep=space, belowskip=5pt]{caption}
\usepackage{hyperref}
\usepackage{minted}
\usepackage{titling}

\usepackage{fancyhdr}
\usepackage[title]{appendix}

\DeclareMathOperator{\sgn}{sgn}

\pagestyle{fancy}
\headheight 24pt
\headsep    12pt
\lhead{Material Property Design}
\rhead{\today}
\fancyfoot[C]{} % hide the default page number at the bottom
\lfoot{}
\rfoot{\thepage}
\renewcommand{\headrulewidth}{0.4pt}
\renewcommand\footrulewidth{0.4pt}
\providecommand{\abs}[1]{\lvert#1\rvert}
\providecommand{\norm}[1]{\lVert#1\rVert}
\providecommand{\dx}{\, \mathrm{d}x}
% \providecommand{\vint}[2]{\int_{#1} \! #2 \, \mathrm{d}x}
% \providecommand{\sint}[2]{\int_{\partial #1} \! #2 \, \mathrm{d}A}
\renewcommand{\div}{\nabla \cdot}
\providecommand{\shape}{\Omega(p)}
\providecommand{\mesh}{\mathcal{M}}
\providecommand{\boundary}{\partial \shape}
\providecommand{\vint}[1]{\int_{\shape} \! #1 \, \mathrm{d}x}
\providecommand{\sint}[1]{\int_{\boundary} \! #1 \, \mathrm{d}A}
\providecommand{\pder}[2]{\frac{\partial #1}{\partial #2}}
\providecommand{\tder}[2]{\frac{\mathrm{d} #1}{\mathrm{d} #2}}
\providecommand{\evalat}[2]{\left.#1\right|_{#2}}
\newcommand{\defeq}{\vcentcolon=}
\newtheorem{lemma}{Lemma}

\makeatletter
\usepackage{mathtools}
\newcases{mycases}{\quad}{%
  \hfil$\m@th\displaystyle{##}$}{$\m@th\displaystyle{##}$\hfil}{\lbrace}{.}
\makeatother

\setlength{\droptitle}{-50pt}
\title{Approach to Material Property Design}
\author{}
% Move date up over where author would be...
\date{\vspace{-24pt} \today}

% BEGIN DOCUMENT
\begin{document}
\maketitle
We outline the basic approach and tools needed to achieve spatially varying
material properties on a single material printer via periodic, patterned
microstructures. We assume for now that we are given a coarse volume mesh,
$\mesh_c$, and that each element has been assigned an elasticity tensor or
material parameters. Ideally, each of these tensors lies in the space of
homogenized tensors spanned by our chosen patterns (this could be enforced by
the tool/optimization used to assign material properties). Our goal is to create
a fine-scale volume mesh, $\mesh_f$, filled with the printer's single,
homogeneous material so that the homogenized behavior of $\mesh_f \cap e$
approximates $e$'s input elasticity tensor for all volume elements $e \in
\mesh_c$.
\section{Overview}
 The general approach is as follows:
\begin{enumerate}
    \setcounter{enumi}{-1}
    \item Read in the target material properties, $m_e$, for each $e \in
        \mesh_c$. These could be in the form of
        \label{step:read}
        \begin{enumerate}
            \item ``painted'' elasticity tensors from a library we know our
                patterns can approximate;
            \item elasticity tensors fit to object deformation measurements (e.g.
                  Matusik's ``shoe paper'', \cite{Bickel2010}); or
            \item elasticity tensors optimizing some objective (minimum
                  compliance, desired deformation, desired force
                  feedback).
        \end{enumerate}
    \item Fit pattern parameters. For each element $e \in \mesh_c$, invert the
        pattern parameters $\to$ material properties map to find parameters,
        $p_e$, for the periodic pattern closest to achieving the desired
        properties.
        \label{step:fit}
        \begin{enumerate}
            \item Use a precomputed pattern parameter to material properties
                lookup table built for our patterns to interpolate an
                approximate inverse. This table could be built by sampling an
                ``$\epsilon \to 0$'' formula, e.g. for sequential laminates
                \cite{allaire2002shape}, or by numerical homogenization \`a la
                \cite{Kharevych2009}.
            \item If we have an explicit ``$\epsilon \to 0$'' formula, run some
                iterations of Newton's method to improve the inverse.
        \end{enumerate}
        Notice that here we are taking advantage of Theorem 2.1.2 from
        \cite{allaire2002shape} to work pointwise; it says that a homogenized
        material field is valid/achievable if and only if its tensor value at
        every point is an elasticity tensor obtained by periodic homogenization.
    \item Generate a printable, fine-scale mesh, $\mesh_f$. Because patterns
        need to ``connect'' across coarse element boundaries, we cannot operate
        on each coarse element in isolation.
        \label{step:mesh}
        \begin{enumerate}
            \item Choose a length-scale, $\epsilon$ (periodic cell size), based
                on printer resolution and printability concerns (thin features,
                etc.).
            \item Subdivide each $e \in \mesh_c$ into cells of this size,
                filling each with a mesh of the periodic pattern specified by
                $p_e$.
            \item Stitch together the meshes (see Section \ref{sec:connect}).
        \end{enumerate}
    \item Fabricate $\mesh_f$.
    \label{step:fabricate}
\end{enumerate}

\section{Validation}
Each stage of the pipeline will introduce errors, and it will be important to
measure and analyze them individually.

From this writeup's perspective, step \ref{step:read} gives us the ground truth:
the final, fabricated object better behave like a FEM simulation of $\mesh_c$.
However, if/when the scope of our project expands to include
compliance/deformation optimization or object duplication, we will need to see
how much this simulation differs from the desired result.

Step \ref{step:fit}'s error can be measured accurately if we have a formula
that maps pattern parameters to homogenized material properties. For the class
of sequential laminates, we have just that: e.g. (2.67) and (2.151) in
\cite{allaire2002shape}. However, if we instead only have a lookup table,
measuring the error is impossible; we do not know the material properties
corresponding to our interpolated guess of pattern parameters. Worse, if our
desired material properties are far from the entries in the lookup table, we do
not know if that is because the lookup table is incomplete or the material is
not achievable.

Step \ref{step:mesh} unfortunately accumulates two errors: one is .

Step \ref{step:fabricate} introduces another likely large error: unpredictable
variations in the 3D printer's material properties. Unfortunately, there's not
much we can do here. Hopefully the error will be small enough that we can make
qualitative comparisons to the ground-truth simulation of $\mesh_c$ and only
present quantitative comparisons in the evaluation of $\mesh_f$ (step
\ref{step:mesh}).

\section{Search for Patterns}

\section{Connecting the Patterns}
\label{sec:connect}

\section{Concerns}
The homogenization theory is all for linear elasticity. If we want to design
large-scale deformation behavior we will probably have.

\bibliographystyle{plain}
\bibliography{References}

\end{document}

