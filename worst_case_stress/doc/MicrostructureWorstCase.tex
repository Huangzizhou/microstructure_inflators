\documentclass[10pt]{article}

\usepackage[latin1]{inputenc}
\usepackage{amsmath, amssymb, amsfonts, amsthm}
\usepackage{upgreek}
\usepackage{amsthm}
\usepackage{fullpage}
\usepackage{graphicx}
\usepackage{cancel}
\usepackage{subfigure}
\usepackage{mathrsfs}
\usepackage{outlines}
\usepackage[font={sf,it}, labelfont={sf,bf}, labelsep=space, belowskip=5pt]{caption}
\usepackage{hyperref}
% \usepackage{minted}
\usepackage{titling}
\usepackage{xifthen}

\usepackage{fancyhdr}
\usepackage[title]{appendix}
\usepackage{float}

\DeclareMathOperator{\sgn}{sgn}
\DeclareMathOperator{\tr}{tr}
\DeclareMathOperator{\diag}{diag}
\renewcommand{\Re}{\operatorname{Re}} \renewcommand{\Im}{\operatorname{Im}}
\allowdisplaybreaks

\pagestyle{fancy}
\headheight 24pt
\headsep    12pt
\lhead{Worst-Case Stress Analysis and Optimization for Microstructures}
\rhead{\today}
\fancyfoot[C]{} % hide the default page number at the bottom
\lfoot{}
\rfoot{\thepage}
\renewcommand{\headrulewidth}{0.4pt}
\renewcommand\footrulewidth{0.4pt}
\providecommand{\abs}[1]{\lvert#1\rvert}
\providecommand{\norm}[1]{\lVert#1\rVert}
\providecommand{\dx}{\, \mathrm{d}x}
% \providecommand{\vint}[2]{\int_{#1} \! #2 \, \mathrm{d}x}
% \providecommand{\sint}[2]{\int_{\partial #1} \! #2 \, \mathrm{d}A}
\renewcommand{\div}{\nabla \cdot}
% We're talking about microstructure shape \omega
\providecommand{\shape}{\omega}
\providecommand{\mesh}{\mathcal{M}}
\providecommand{\boundary}{\partial \shape}
\providecommand{\bdry}{\boundary}
\providecommand{\vint}[3][x]{\int_{#2} \! #3 \, \mathrm{d}#1}
\providecommand{\sint}[3][x]{\int_{#2} \! #3 \, \mathrm{d}A(#1)}
\providecommand{\pder}[2]{\frac{\partial #1}{\partial #2}}
\providecommand{\tder}[2]{\frac{\mathrm{d} #1}{\mathrm{d} #2}}
\providecommand{\evalat}[2]{\left.#1\right|_{#2}}
\renewcommand{\vec}[1]{{\bf #1}}

\usepackage{prettyref}
\newrefformat{sec}{Section~\ref{#1}}
\newrefformat{tbl}{Table~\ref{#1}}
\newrefformat{fig}{Figure~\ref{#1}}
\newrefformat{chp}{Chapter~\ref{#1}}
\newrefformat{eqn}{\eqref{#1}}
\newrefformat{set}{\eqref{#1}}
\newrefformat{alg}{Algorithm~\ref{#1}}
\newrefformat{apx}{Appendix~\ref{#1}}
\newcommand\pr[1]{\prettyref{#1}}

\def\normal{\hat{n}}
\def\n{{\bf n}}
\def\x{\vec{x}}
\def\X{\vec{X}}
\def\y{\vec{y}}
\def\z{\vec{z}}
\def\u{\vec{u}}
\def\w{\vec{w}}
\def\p{\vec{p}}
\def\q{\vec{q}}
\def\ue{\vec{u}^\e}
\def\strain{\varepsilon}
\def\stress{\sigma}
\def\d{\, \mathrm{d}}
\def\R{\, \mathbb{R}}
\def\L{\mathcal{L}}
\def\flat{\mathscr{F}}

\DeclareMathOperator*{\argmin}{argmin}
\DeclareMathOperator*{\sym}{sym}
\newcommand{\defeq}{\vcentcolon=}
\newtheorem{lemma}{Lemma}

\makeatletter
\usepackage{mathtools}
\newcases{mycases}{\quad}{%
  \hfil$\m@th\displaystyle{##}$}{$\m@th\displaystyle{##}$\hfil}{\lbrace}{.}
\makeatother

\setlength{\droptitle}{-50pt}
\title{Worst-Case Stress Analysis and Optimization for Microstructures}
\author{Julian Panetta}

% BEGIN DOCUMENT
\begin{document}
\maketitle

The low dimensional space of macroscopic stresses that a microstructure can
experience (6 in 3D, 3 in 2D) leads to an efficient formulation of the
worst-case stress appearing at a point in the microstructure. We present some
variants of this formulation and describe how to shape-differentiate them
accurately in the continuous and discrete setting.

\section{Macroscopic to Microscopic Stress}
We can use the fluctuation displacements (homogenization cell problem solutions)
to determine the stress at a point $\x \in \Omega$ under a particular macroscopic strain
scenario, $\sigma$. Using base elasticity tensor $C^\text{base}$, homogenized
compliance tensor $S^H$, and a rank 4 tensor ``G'' mapping macroscopic strains to true microscopic strains at $\x$,
$$
G_{ijkl}(\x) = [\strain(\w^{kl})(\x) + e^{kl}]_{ij} \quad (\w^{kl} \text{ is the } kl^\text{th} \text{ fluctuation displacement}),
$$
we can compute the true microscopic stress at point $\x$ under $\sigma$:
\begin{equation}
\label{eqn:macro_micro}
C^\text{base} : G(\x) : S^H : \sigma \equiv F : \sigma,
\end{equation}
where we've defined $F = C^\text{base} : G(\x) : S^H$ to be the linear
macroscopic-stress-to-stress-at-$\x$ map. Notice that $F$ has minor (but not major) symmetries.

\section{Worst-Case Stress Definitions}
We can then write the worst-case max stress at $\x$:
$$
\max_\sigma \lambda_\text{max}(F : \sigma) = \max_\sigma \max_{\norm{\n}^2 = 1} \n^T [F : \sigma] \n.
$$
We can also express the worst-case squared Frobenius norm stress at $\x$ as
$$
\max_\sigma \norm{F : \sigma}_F^2 =
\max_\sigma \sigma : F^T : F : \sigma,
$$
and the worst-case von Mises stress as:
$$
\max_\sigma \sigma : F^T : V^T : V : F : \sigma,
$$
where $V$ extracts the ``von Mises component,'' i.e. ($\sqrt{\frac{3}{2}}$
times) the deviatoric stress component.

\subsection{Macroscopic Stresses}
We next must decide what subspace of macroscopic stresses to consider. There
are two natural constraint choices: $\lambda_\text{max}(\sigma) = 1$ or
$\norm{\sigma}_F^2 = \sum_i \lambda_i(\sigma)^2 = 1$. They both bound
the traction that is applied to the cell interfaces, but the Frobenius
norm is easier to work with and arguably measures stress magnitude more intuitively:
e.g. it treats the identity tensor as a higher-stress state than
$\begin{pmatrix}1 & 0 \\ 0 & 0\end{pmatrix}$.

\subsection{Worst-case Max Stress}
Now we attempt to perform the maximization:
\begin{equation}
\label{eqn:worstEigenvalueForm}
\lambda_\text{worst}(\x) = \max_{\norm{\sigma}^2_F = 1} \max_{\norm{\n}^2 = 1} \n^T [F : \sigma] \n,
\end{equation}
whose corresponding unconstrained problem is
$$
\lambda_\text{worst}(\x) = \max_{\sigma} \max_{\n} \min_{\lambda_\sigma} \min_{\lambda_\n} \L(\sigma, \n, \lambda_\sigma, \lambda_\n)
$$
with Lagrangian
\begin{align*}
\L(\sigma, \n, \lambda_\sigma, \lambda_\n) &= \n^T [F : \sigma] \n + \lambda_\sigma \left(\norm{\sigma}^2_F - 1\right)  + \lambda_\n \left(\norm{\n}^2 - 1\right) \\
                                           &= n_i F_{ijkl} \sigma_{kl} n_j + \lambda_\sigma \left(\sigma_{ij}\sigma_{ij} - 1\right)  + \lambda_\n \left(n_i n_i - 1\right) 
\end{align*}
Stationarity with respect to $\sigma$ and $\n$ requires:
\begin{align}
\label{eqn:stationarySigma}
\left[\pder{\L}{\sigma}\right]_{pq} &= n_i F_{ijpq} n_j + 2 \lambda_\sigma \sigma_{pq} = 0 \quad \left(\pder{\L}{\sigma} = [\n \n^T] : F + 2 \lambda_\sigma \sigma = 0\right) \\
\label{eqn:stationaryN}
      \pder{\L}{\n} &= 2 [F : \sigma] \n + 2 \lambda_\n \n = 0.
\end{align}
Stationarity with respect to $\n$ of course just reveals that $\n$ must be an
eigenvector of $F : \sigma$. Stationarity with respect to $\sigma$ is more
interesting. It shows that, for arbitrary $\n$, the greatest principal stress is
achieved by:
$$
\sigma^* = - \frac{[\n \n^T] : F}{2 \lambda_\sigma},
$$
and consequently the worst-case macroscopic stress load must take this form.

Since $\sigma^*$ must have unit Frobenius norm, we can solve for
$\lambda_\sigma$ to get the precise load corresponding to $\n$:
\begin{gather*}
1 = \sigma^* : \sigma^*
  = \frac{([\n {\n}^T] : F) : ([\n {\n}^T] : F)}{\left(2 \lambda_\sigma\right)^2}
  = \frac{[\n {\n}^T] : F : F^T : [\n {\n}^T] }{\left(2 \lambda_\sigma\right)^2} \\
\Longrightarrow
    2 \lambda_\sigma = \pm \sqrt{[\n {\n}^T] : F : F^T : [\n {\n}^T]} \\
\Longrightarrow
    \sigma^* = \pm \frac{[\n \n^T] : F}{\sqrt{[\n {\n}^T] : F : F^T : [\n {\n}^T]}}.
\end{gather*}
To get the greatest tensile stress---instead of compressive---we choose the
positive sign.

We can now rewrite \pr{eqn:worstEigenvalueForm} as a maximization over $\n$
only:
$$
\lambda_\text{worst}(\x) = \max_{\norm{\n}^2 = 1} \n^T [F : \frac{[\n \n^T] : F}{\sqrt{[\n {\n}^T] : F : F^T : [\n {\n}^T]}}] \n,
                         = \frac{[\n {\n}^T] : F : F^T : [\n {\n}^T]}{\sqrt{[\n {\n}^T] : F : F^T : [\n {\n}^T]}},
$$
or
$$
\boxed{
\lambda_\text{worst}^2(\x) = \max_{\norm{\n}^2 = 1} [\n {\n}^T] : F : F^T : [\n {\n}^T].
}
$$
This is now a symmetric rank 4 tensor eigenvalue problem: it's easy to show with Lagrange multipliers that it is maximized by a $\n^*$ s.t.
$$
\n^* \cdot T^\text{max} : [\n^* {\n^*}^T] = \lambda_\text{worst}^2 \n^*,
$$
where $T^\text{max} = F : F^T$.
It can be solved efficiently either with equality constrained minimization or by
using an algorithm specifically designed for solving this type of eigenvalue
problem, which has recently received attention in the numerical linear algebra
community (TODO: insert citations).

\section{Worst-case Frobenius Norm/von Mises Stress}
Maximizing the microscopic stress' Frobenius norm at a point (or the
closely-related von Mises stress) is even simpler. Again taking $F$ to be the
linear macro-to-micro stress map at point $x$, we have the problem:
$$
\max_{\norm{\sigma}_F^2 = 1} \norm{F : \sigma}_F^2 = 
\max_{\norm{\sigma}_F^2 = 1} \sigma : T : \sigma,
$$
where $T$ is either $T^\text{Frob} \defeq F^T : F$ for the Frobenius norm case
or $T^\text{vonMises} \defeq F^T : V^T : V : F$ for the von Mises case. (note:
both of these differ from $T^\text{max}$).
This is now a generalized matrix eigenvalue problem for the flattened form of
$T^\text{Frob}$:
$$
\min_{\flat(\sigma)^T \mathcal{D} \flat(\sigma) = 1} \flat(\sigma)^T
\mathcal{D} \flat(T) \mathcal{D} \flat(\sigma)
$$
where $\flat(\cdot)$ represents the flattening operator and $\mathcal{D}$ is
the ``shear-doubling'' matrix (see the Tensor Flattening write-up). Since
$\mathcal{D}$ is a positive diagonal matrix, it's easy to translate this into
an ordinary, inexpensive $6 \times 6$ matrix eigenvalue problem ($3 \times 3$
in 2D).

\section{Minimizing Worst-case Stress}
All three stress measures (principal stress, Frobenius norm, von Mises), can be
minimized in the same framework. In the following, $T$ is chosen from $\{
    T^\text{max},\ T^\text{Frob},\ T^\text{vonMises}\}$, and $\sigma^*$ is
computed accordingly. In the max-stress case, that means $\sigma^* = \n^* { \n^*
}^T$ where $\n$ is a rank 4 tensor eigenvector, and in the other cases, $\sigma^*$
is found by flattenting $T$, computing a matrix eigenvector, and unflattening.

\subsection{Derivatives of Eigenvalues}
\label{sec:eigen_deriv}
The following nice property of eigenvalue derivatives simplifies the
minimization of microstructure worst-case stress. Assume matrix $A$ has a
non-repeated maximum eigenvalue $\lambda$ with corresponding unit eigenvector
$v$.
Then
$$
\dot \lambda = \tder{}{t} (v^T A v ) = \dot v^T A v + v^T \dot A v + v^T A \dot v = \lambda \cancel{\dot v^T v} + v^T \dot A v + \lambda \cancel{v^T \dot v} = v^T \dot A v,
$$
where we used the fact that $v(t)$ is a unit vector.
Similarly, assume rank four tensor $T$ has a non-repeated maximum eigenvalue
$\lambda$ with corresponding unit eigenvector $n$. Then
\begin{align*}
    \dot \lambda &= \tder{}{t} (T_{ijkl} n_i n_j n_k n_l) = \dot{T}_{ijkl} n_i n_j n_k n_l + \dot n_i (T_{ijkl} n_j n_k n_l) + \dots  \\
    &= \dot{T}_{ijkl} n_i n_j n_k n_l + 4 \lambda \cancel{\dot n_i n_i} = \dot{T}_{ijkl} n_i n_j n_k n_l.
\end{align*}
In other words, we can write the eigenvalues as $v^T A v$ and $T_{ijkl} n_i n_j n_k n_l$ and
then treat the eigenvectors as constants when differentiating. In the context
of worst-case stress, this means {\em we can treat the worst-case macroscopic load
for each point as constant when shape-differentiating worst-case stress}.

In practice, the elements with high stress---the ones on which we need accurate
derivatives---have one dominant maximum eigenvalue, so we never
need to worry about a repeated eigenvalue breaking differentiability (I have
visualizations showing this, if needed).

\subsection{Global Objective}
We consider a global objective function obtained from the worst-case stress
field $s(x)$. For notational simplicity we consider only the worst-case
Frobenius norm stress minimization, but the others are identical (up to derivatives of $T$):
$$
J(s) = \vint{\omega}{j(s(x))} = \vint{\omega}{j(\sigma^*(x) : T^\text{Frob}(x) : \sigma^*(x))}
$$
For instance, this could be the $p^\text{th}$ power of our stress measure,
$$
j(s) = s^p.
$$
To shape-differentiate $J$, it will be necessary to know how $j$
changes at each $x$ when the fluctuation strains change.
As discussed in \pr{sec:eigen_deriv}, 
the $\sigma^*$ in $s(x)$ can be considered constant wrt the shape, but $T^\text{Frob}(x)$ is a function of
the fluctuation strains via \pr{eqn:macro_micro}, which change with the shape.
Furthermore, though homogenized elasticity tensor $C^H$ is
technically a function of the fluctuation strains, it simplifies our derivation
to view $C^H$ as an independent parameter of $s$ since we already know how to compute
its shape derivative efficiently.
Representing these relationships explicitly, we write:
$$
j(s(x)) := j(s(\strain^{pq}, C^H, x)),
$$
where $\strain^{pq}$ expands to six strain field arguments in 3D.
$$
\delta j = (j')\pder{s}{\strain^{kl}} : \delta \strain^{kl} + (j')\pder{s}{C^H} :: \delta C^H.
$$
For now, we define
$$
\tau^{kl} \defeq (j')\pder{s}{\strain^{kl}}, \quad \gamma \defeq (j')\pder{s}{C^H}
\quad \quad \Longrightarrow \quad \quad \delta j = \tau^{kl} : \delta \strain^{kl} + \gamma : \delta C^H,
$$
and defer explicit computation of $\tau^{kl}$ and $\gamma$ to \pr{sec:tau}.

\subsection{Optimization}
We wish to minimize the worst-case stress objective over microstructures
achieving a particular homogenized tensor $C^*$.
This problem takes the form:
$$
\min_{\substack{\omega \text{ admissible} \\ C^H(\omega) = C^*}} \vint{\omega}{j(\sigma^*(x) : T(x) : \sigma^*(x))}.
$$
In our examples, we actually use the objective,
$$
\min_{\substack{\omega \text{ admissible} \\ C^H(\omega) = C^*}} \vint{\omega}{(\sigma^*(x) : T(x) : \sigma^*(x))^{\frac{p}{2}}},
$$
which minimizes the $L_p$ norm of stress; notice that $T$'s quadratic form computes the {\em squared} worst-case stress measure.

\subsection{Shape Derivative}
We wish to use gradient-based optimization, so we need derivatives of the
objective and constraints. Since we optimize over shapes, in particular we need
a shape derivative.

There are two approaches (forward and adjoint) to shape-differentiating $J$,
i.e. finding the change in $J$ when microstructure boundary $\partial \omega$
is advected by velocity field $v$. Both start with the equation:
\begin{align}
    \notag
        dJ[v] &= \sint{\bdry}{(v \cdot \normal) j} + \vint{\shape}{\tau^{kl} : \strain(\dot \w^{kl}[v]) + \gamma :: dC^H[v]} \\
              &= \sint{\bdry}{(v \cdot \normal) j} + \underbrace{\vint{\shape}{\tau^{kl} : \strain(\dot \w^{kl}[v])}}_{I} + \left(\vint{\shape}{\gamma}\right) :: dC^H[v].
    \label{eq:generic_shape_derivative}
\end{align}
(This equation follows from Reynold's Transport Theorem, and $\dot \w^{kl}[v]$
is an Eulerian derivative wrt ``time'' ($\pder{}{t}$) under the advection velocity $v$).
The first and third terms are straightforward to evaluate, but the middle
integral ``$I$'' involves the problematic term $\dot{\w}^{kl}[v]$, which measures
how the fluctuation displacements change when perturbing the shape with
velocity field $v$.

The ``forward'' sensitivity analysis determines, for a particular velocity
field $v$, the change in fluctuation displacements $\dot{\w}^{kl}[v]$ and
plugs this into \pr{eq:generic_shape_derivative}. This method involves solving
a set of cell problems for each velocity field $v$, so it is a bad choice when
the space of velocity fields is high-dimensional. But it is good when we only
have a few shape parameters and, e.g., want to know the change in worst-case
stress at every point (as opposed to the change in single objective function,
$J$).

For only one or a few objective functions, the adjoint method is preferable.
This approach solves a set of ``adjoint cell problems'' per objective function,
and then rewrites the problematic integral of \pr{eq:generic_shape_derivative}
as a boundary integral not involving $\dot{\w}^{kl}$:
$$
I = \sint{\bdry}{(v \cdot \normal) f}
$$
where $f$ is a scalar field independent of $v$ that involves the adjoint
solutions. The value of $I$ for a particular velocity field $v$ can then be
computed inexpensively by evaluating these integrals. Since the other two terms
of $dJ[v]$ are already boundary integrals in this form (after substituting
in $dC^H$), the objective's shape derivative can be written as:
$$
dJ[v] = \sint{\bdry}{(v \cdot \normal) g},
$$
where again $g$ is a scalar field independent of $v$. This is particularly nice
because $g$ (the Riesz representative of $dJ[\cdot]$) represents the
steepest descent normal velocity field, which can be visualized and may suggest
how we should expand the parametric design space.

\subsubsection{Forward Version}
For a particular $v$, we can determine an equation for $\dot{\w}^{kl}[v]$ by
differentiating the weak form of the $kl^{th}$ cell problem.
To simplify the derivation, we formulate the cell problem on a flat torus; this
implicitly applies periodic boundary conditions on the space of admissible
fluctuation displacements and their derivatives. (With some extra work, we
could instead enforce the periodic boundary conditions with Lagrange
multipliers to reach the same results.)
Similarly, we omit the details of the
no-rigid-translation constraint, constraining all trial/test
functions to have no rigid translation. This is okay since the no rigid
translation constraint is independent of the domain's shape.

The cell problem's weak form is:
\begin{equation}
\label{eqn:weak_cell}
    \Longrightarrow \quad
    \vint{\omega}{\strain(\phi) : C : [\strain(\w^{kl}) + e^{kl}]}
    = 0
    \quad \forall \phi \\
\end{equation}
Where $\w^{kl}$ and $\phi$ are periodic vector fields on the unit cell $Y$.
Differentiating both sides of this equation using Reynold's Transport Theorem,
we find:
\begin{align}
\label{eqn:weak_forward}
\Aboxed{
\sint{\partial \omega}{(v \cdot \normal) \bigg( \strain(\phi) : C : [\strain(\w^{kl}) + e^{kl}]\bigg)}
+ \vint{\omega}{\strain(\phi) : C : \strain(\dot \w^{kl}[v])}
&= 0 \quad \forall \phi
}
\end{align}
This is the weak from of a cell problem for $\dot \w^{kl}[v]$ with
shape-velocity-dependent load given by the boundary integral. Once we've solved
this equation for each $\dot \w^{kl}[v]$, we can compute
\pr{eq:generic_shape_derivative} easily.

\subsubsection{Adjoint Version}
We determine the adjoint equations by noticing the following: suppose we can
find an ``adjoint solution'' $p^{kl}$ from the same space as $\phi$
(i.e., a periodic test function for original PDE) such that
\begin{equation}
\boxed{
    \label{eqn:weak_adjoint}
    \vint{\omega}{\tau^{kl} : \strain(\psi)} =
    \vint{\omega}{\strain(p^{kl}) : C : \strain(\psi)} \quad \forall \psi,
}
\end{equation}
where $\psi$ is from the same space as $\dot \w^{kl}$ (i.e., a periodic trial
function for original PDE). Then we can use $p^{kl}$ to compute integral $I$ as follows:
\begin{gather*}
\begin{alignedat}{2}
  I &= \vint{\omega}{\tau^{kl} : \strain(\dot \w^{kl})} \\
    &= \vint{\omega}{\strain(p^{kl}) : C : \strain(\dot \w^{kl})} && \quad \quad \text{by \pr{eqn:weak_adjoint}} \\
    &= -\sint{\partial \omega}{(v \cdot \normal) \bigg( \strain(p^{kl}) : C : [\strain(\w^{kl}) + e^{kl}]\bigg)}. && \quad \quad \text{by \pr{eqn:weak_forward}} \\
\end{alignedat}
\end{gather*}
The second line follows by substituting $\dot \w^{kl}$ for $\psi$ in \pr{eqn:weak_adjoint}, and the third
by substituting $p^{kl}$ for $\phi$ in \pr{eqn:weak_forward}.

Using this formula, our full shape derivative can be computed efficiently as:
\begin{equation}
\label{eqn:cts_shape_der}
\boxed{
dJ[v] =
\sint{\partial \omega}{\bigg(j -
\strain(p^{kl}) : C : [\strain(\w^{kl}) + e^{kl}]\bigg)v \cdot \normal}
+ \left(\vint{\shape}{\gamma}\right) :: dC^H[v],
}
\end{equation}
summing over $kl$.

We recognize \pr{eqn:weak_adjoint} as the weak form of the adjoint cell problem
PDE:
\begin{gather*}
    \boxed{\begin{alignedat}{2}
        -\div \stress(p^{kl}) &= -\div \tau^{kl} &&\quad \text{in } \omega, \\
        \stress(p^{kl}) \normal &= \tau^{kl} \normal &&\quad \text{on } \partial \omega, \\
            p^{kl} \text{ periodic}, & && \vint{\omega}{p^{kl}} = 0.
    \end{alignedat} }
\end{gather*}

\subsection{Computing \texorpdfstring{$\tau^{kl}$}{Tau\textasciicircum kl} and \texorpdfstring{$\gamma$}{Gamma}}
\label{sec:tau}
First we compute the rank-two tensor field $\tau^{kl}$ expressing the
derivative of objective integrand $j$ with respect to fluctuation strain
$\strain^{kl}$ (considering $C^H$ and thus $S^H$ constant). Note that we are
considering the Frobenius norm case ($T^\text{Frob}$); the others will differ
slightly in this computation.
\begin{align*}
    \tau^{kl}_{ij} \defeq& j' \left. \pder{}{\strain^{kl}_{ij}} \middle[ \sigma^* : F^T : F : \sigma^* \right] \\
    =& j'  \sigma^* : \left[\left(\pder{}{\strain^{kl}_{ij}} F^T \right) : F + F^T : \pder{}{\strain^{kl}_{ij}}F \right] : \sigma^* \\
    =& 2 j'  \sigma^* : F^T : \left( \pder{}{\strain^{kl}_{ij}}F \right) : \sigma^*.
\end{align*}
Here, $^T$ denotes major transposition, and we used the fact that a tensor
and its transpose give the same quadratic form.

By definition \pr{eqn:macro_micro},
\begin{align*}
    \pder{}{\strain^{kl}_{ij}} F_{abcd} &= C^\text{base}_{abef} S^H_{ghcd} \pder{}{\strain^{kl}_{ij}} (\strain^{gh}_{ef} + \cancelto{0}{e^{gh}_{ef}}) \\
                                        &= C^\text{base}_{abef} S^H_{ghcd} \delta_{gk} \delta_{hl} \delta_{ei} \delta_{fj} \\
                                        &= C^\text{base}_{abij} S^H_{klcd}.
\end{align*}
Plugging this in and simplifying, we arrive at:
\begin{equation*}
        \tau^{kl}_{ij} = 2 j' [C^\text{base} : F : \sigma^*]_{ij} [S^H : \sigma^*]_{kl}.
\end{equation*}
or in index free notation:
\begin{equation}
\label{eqn:tau}
\boxed{
    \tau^{kl} = (2 j' C^\text{base} : F : \sigma^*) \otimes (S^H : \sigma^*).
}
\end{equation}

Next, we compute the rank-four tensor field $\gamma$ expressing the partial derivative
of objective integrand $j$ with respect to the homogenized elasticity tensor $C^H$.
\begin{align*}
    \delta j &=  2 j' \sigma^* : F^T : C^\text{base} : G : dS^H : \sigma^* \\
             &=  2 j' \sigma^* : F^T : C^\text{base} : G : (-S^H : dC^H : S^H) : \sigma^* \\
             &= (-2 j' \sigma^* : F^T : F) : dC^H : (S^H : \sigma^*) \\
             &= \left[(-2 j' \sigma^* : F^T : F) \otimes (S^H : \sigma^*)\right] :: dC^H
\end{align*}
Thus,
\begin{equation}
\label{eqn:gamma}
\boxed{
    \gamma = (-2 j' F^T : F : \sigma^*) \otimes (S^H : \sigma^*).
}
\end{equation}

\section{Discrete Formulation}
There are two separate discretizations to consider: discretization of the cell
problem PDEs and discretization of the worst-case stress objective integral.
We discretize the cell problems with quadratic triangle/tetrahedral
straight-edged (subparametric) finite elements, which we found essential for
accurate stress and homogenized tensor calculation.

Notice that the stress tensor is not constant in each element, but rather
piecewise linear. Thus, the worst-case stress integrand is highly nonlinear
over each element, and we must choose a quadrature scheme to approximately
integrate it. We obtained good results with just piecewise constant quadrature,
which is equivalent to first averaging the stress over each element before
computing per-element worst-case stress quantities. A higher order quadrature
rule could be used, in which case worst-case stresses must be computed at each
quadrature point.

\subsection{Discrete Objective}
In the discrete setting (under piecewise constant quadrature) the objective is:
$$
J_d = \sum_{e} j(s[e]) V_e,
$$
where $s[e]$ is the worst-case stress on element $e$ and $V_e$ is 
the element volume. We compute $s[e]$ by solving the eigenvalue problem for
piecewise constant rank-4 tensor field
$$
T_e = F^T_e : F_e \quad \quad (F_e = C^\text{base} : G_e : S^H),
$$
where $G_e$ is formed by evaluating the fluctuation strains at the quadrature
point (i.e. averaging them over the element).

Identically to the continuous case, we can determine the change in the
integrand on element $e$ due to the changing (element-averaged) fluctuation
strains and homogenized tensors. These sensitivities are encoded by piecewise
constant quantities $\tau^{kl}_e$ and $\gamma_e$ respectively, computed by
\pr{eqn:tau} and \pr{eqn:gamma}.

\subsection{Discrete Shape Derivatives}
Since we consider straight-edged finite elements, the perturbation velocity
$v$ is a piecewise linear vector field and is represented as a perturbation
vector on each mesh vertex:
$$
v = \sum_{i} \lambda_i \delta \q_i\ ,
$$
where $\lambda_i$ is vertex $i$'s linear shape function (barycentric
coordinates) and $\delta \q_i$ is its perturbation.

Unfortunately, the most straight-forward approach of plugging this piecewise
linear $v$ into a discretized version of \pr{eqn:cts_shape_der} leads to wildly
inaccurate shape derivatives for the high $L_p$ norms needed to prevent stress
concentrations. This is despite that particular form of shape derivative
appearing in the literature. For instance, \cite{Allaire2008909} uses an analogous
formula in the level set optimization framework for minimum stress problems,
possibly explaining why it only considers $L_2$ and $L_5$ norms.

It turns out that the Reynold's Transport Theorem and the Eulerian derivatives
used in \ref{eqn:5} are the root of the error. In retrospect, this makes
sense: though in the continuous the shape derivatives are determined only by
the boundary perturbation, in the discrete case there will be a nonzero
contribution from the internal nodes. For instance, perturbing the
boundary-adjacent vertices away from the boundary in the regions of high stress
will artificially reduce the stress computed on these elements. Also, it is
most natural to work with material derivatives in the discrete setting (instead
of Eulerian derivatives), the finite element fields are tied to the moving mesh
nodes (material points).

Mathematically, the error is introduced by an integration by parts used to
derive Reynold's Transport Theorem that does not hold in the discrete setting.
Starting at the last step of the derivation whose discretization still holds:
\begin{align}
\notag
dJ[v]
&= \vint{\shape}{j \div v + \tau^{kl} : D[\strain(\w^{kl})] + \gamma :: dC^H[v]} \\
&= \vint{\shape}{j \div v + \underbrace{\tau^{kl} : D[\strain(\w^{kl})]}_{II}} + \left(\vint{\shape}{\gamma}\right) :: dC^H[v],
\label{eqn:discretizable_dJ}
\end{align}
where $D[\cdot]$ denotes the material derivative $\left(\pder{}{t} + v \cdot
\nabla\right)$. Again, $II$ is the difficult term to compute.

Notice that this formula involves quantities over all of $\omega$, not just the boundary.
The $\div v$ quantity measures how the material at each integration point
dilates under advection by $v$, and $D[\strain(\w^{kl})]$ computes how the
fluctuation strain at each material point changes.

Notice the subtlety that $D[\strain(\w^{kl})]$ does not look like the strain of
a finite element trial function---and thus cannot be discretized directly in
our FEM framework---because $D[\cdot]$ and $\strain(\cdot)$ do not commute.
However, the following identity holds for any linear combination of shape
functions, $\phi$:
$$
D[\strain(\phi)] = \strain(D[\phi]) - \sym(\nabla \phi \nabla v),
$$
where we use $\nabla$ applied to a vector field to denote the Jacobian (not its
transpose).

Thus:
$$
II = \tau^{kl} : D[\strain(\w^{kl})]
   = \tau^{kl} : \left(\strain(D[\w^{kl}]) - \sym(\nabla \w^{kl} \nabla v) \right).
$$
\subsubsection{Discrete Forward Sensitivity of \texorpdfstring{$\w^{kl}$}{w\textasciicircum kl}}
We can determine $D[\w^{kl}]$, the material derivative of the fluctuation
displacements, by an accurate differentiation of the weak form
\pr{eqn:weak_cell}. First, we define microscopic stress $\sigma^{kl} \defeq C :
\left[\strain(\w^{kl}) + e^{kl}\right]$ to simplify notation. Then,
differentiating both sides of the weak form:
\begin{align}
    \notag
\forall \phi \quad 0
   = &
    \vint{\shape}{\left(\strain(\phi) : \sigma^{kl} \right) \div v +
             D\left[\strain(\phi) : C : \left[\strain(\w^{kl}) + e^{kl}\right] \right]}
\notag
\\ = &
    \vint{\shape}{\left(\strain(\phi) : \sigma^{kl} \right) \div v +
             D[\strain(\phi)] : \sigma^{kl}  +
             \strain(\phi) : C : D[\strain(\w^{kl})]}
\notag
\\ = &
    \vint{\shape}{\left(\strain(\phi) : \sigma^{kl} \right) \div v
    + \left[\strain(\cancelto{0}{D[\phi]}) - \sym(\nabla \phi \nabla v)\right] : \sigma^{kl} 
    + \strain(\phi) : C : \left[\strain(D[\w^{kl}]) - \sym(\nabla \w^{kl} \nabla v)\right]}
\notag
\\ = &
    \vint{\shape}{\left(\strain(\phi) : \sigma^{kl} \right) \div v
    -  \sym(\nabla \phi \nabla v) : \sigma^{kl} 
    + \strain(\phi) : C : \left[\strain(D[\w^{kl}]) - \sym(\nabla \w^{kl} \nabla v)\right]}
.
\label{eqn:discretizable_weak_diff}
\end{align}
This is the weak form of a PDE solving for $D[\w^{kl}]$, which can be
discretized in the straight-forward way: as a vector holding the material
derivative of $\w^{kl}$ at each mesh node. Notice that $D[\phi] = 0$ for all
test functions because the test functions for straight-edged finite elements
are functions of the mesh's barycentric coordinate functions and thus are tied
to material points (i.e. their values advect with the mesh and have zero
material derivative).

\subsubsection{Discrete Forward Sensitivity of $J_d$}
Once the discrete $D[\w^{kl}]$ field is known, it is straightforward to compute
$dJ_d[v]$: the exact discrete derivative is computed by plugging into
\pr{eqn:discretizable_dJ} all discrete fields (piecewise constant $j$, $\tau^{kl}$, and $\gamma$;
piecewise linear $v$; and piecewise quadratic $D[\w^{kl}]$) and using exact
quadrature.

To see why, notice that:
\begin{align*}
d{J_d}[v] &= \sum_{e} j'(s[e]) \tder{s[e]}{t} V_e + j(s[e])\tder{V_e}{t} \\
          &= \sum_{e} \left( \tau^{kl}_e : \overline{D[\strain(w^{kl})]} + \gamma_e :: dC^H[v] + j(s[e])\frac{1}{V_e}\tder{V_e}{t}\right) V_e,
\end{align*}
where $\frac{1}{V_e} \tder{V_e}{t}$ is computed exactly by the piecewise
constant field $\div v$, whose value on element $e$ is:
\begin{equation}
[\div v]_e = \div \sum_{i \in e} \lambda_i \delta \q_i\ ,
       % = \sum_{i} \nabla \lambda_i \cdot \delta \q_i + \lambda_i \cancelto{0}{\div \delta \q_i}
       = \sum_{i \in e} \nabla \lambda_i \cdot \delta \q_i
       = \sum_{i \in e} \frac{ \nabla_{q_i} V_e }{V_e} \cdot \delta \q_i = \frac{1}{V_e} \tder{V_e}{t},
\label{eqn:discrete_div_v}
\end{equation}
and $\overline{D[\strain(w^{kl})]}$ is the average material derivative of
strain over the element (i.e. the material derivative evaluated at the
quadrature point):
$$
\overline{D[\strain(w^{kl})]} = \frac{1}{V_e} \vint{e}{D[\strain(w^{kl})]}.
$$
Thus, finally
\begin{align*}
d{J_d}[v] &= \sum_{e} \left(\frac{1}{V_e}\vint{e}{\tau^{kl}_e : D[\strain(w^{kl})]} + \gamma_e :: dC^H[v] + j(s[e]) [\div v]_e\right) V_e \\
          &= \vint{\omega}{\tau^{kl} : D[\strain(w^{kl})] + j \div v} + \left(\vint{\omega}{\gamma}\right) :: dC^H[v],
\end{align*}
where we used the fact $\tau^{kl}_e$ and $\gamma_e$ are constant on each
element. Notice that this is indeed equivalent to \pr{eqn:discretizable_dJ}.

\subsubsection{Discrete Adjoint Sensitivity}
If we instead want an explicit representation of the differential form
accepting the perturbation velocity fields on $\omega$ and outputting a change
in the objective, we must apply the adjoint method.

The adjoint equations turn out to be identical to \pr{eqn:weak_adjoint} due to
the similarity of integrals $I$ and $II$; simply substitute $D[\w^{kl}]$ for
$\dot \w^{kl}$ in the derivation. However, once we have the adjoint solutions
$p^{kl}$, the exact discrete gradient differs from \pr{eqn:cts_shape_der}.
Instead, we derive it by computing $II$ as follows:
First, substitute $D[\w^{kl}]$ for $\psi$ in \pr{eqn:weak_adjoint} to determine:
\begin{align*}
II &= \vint{\shape}{\tau^{kl} : \left(\strain(D[\w^{kl}]) - \sym(\nabla \w^{kl} \nabla v) \right)} \\
    &= \vint{\shape}{\strain(p^{kl}) : C : \strain(D[\w^{kl}]) - \tau^{kl} : \sym(\nabla \w^{kl} \nabla v)}
\end{align*}
Next, substitute $p^{kl}$ for $\phi$ in \pr{eqn:discretizable_weak_diff} to rewrite the first integrand again:
\begin{align*}
    II &= \vint{\shape}{
    -\left[\strain(p^{kl}) : \sigma^{kl} \right] \div v
     + \sym(\nabla p^{kl} \nabla v) : \sigma^{kl}
 % \\ &\quad \quad \quad
     + \strain(p^{kl}) : C : \sym(\nabla \w^{kl} \nabla v)
     - \tau^{kl} : \sym(\nabla \w^{kl} \nabla v)}.
\end{align*}
Finally, the full discrete shape derivative is evaluated as:
\begin{align*}
dJ_d[v]
    &= \vint{\shape}{\left[j -\strain(p^{kl}) : \sigma^{kl} \right] \div v
     + (\nabla p^{kl} \nabla v) : \sigma^{kl}
     + (\strain(p^{kl}) : C - \tau^{kl}) : (\nabla \w^{kl} \nabla v)}
% \\ &
    + \left(\vint{\shape}{\gamma}\right) :: dC^H[v],
\end{align*}
after substituting in the discrete fields. Notice that we could drop the
symmetrization operator $\sym(\cdot)$ since its output is always double
contracted with a symmetric tensor.

\subsubsection{Discrete Differential Form}
It is convenient to express $dJ_d[v]$ as an explicit inner product with
the per-vertex perturbation vector field $\delta \q$. To do this, we
must re-express the terms involving $v$ in terms of $\delta \q$. The easiest is
$\div v$ which, as we already showed in \pr{eqn:discrete_div_v}, is just
$\sum_m \nabla \lambda_m \cdot \delta \q_m$.

The terms like $\tau^{kl} : (\nabla p^{kl} \nabla v)$ are slightly trickier.
Recall $\nabla$ represents the Jacobian when applied to vectors in this
write-up (rather than its transpose). Thus,
$$
\nabla v = \sum_m \delta \q_m \otimes \nabla \lambda_m.
$$
We can write $\nabla p^{kl}$ in terms of each {\em scalar}-valued finite element shape
function $\varphi_n$ and its {\em vector}-valued coefficient $\vec{p}^{kl}_n$ as:
$$
\nabla p^{kl} = \sum_n \vec{p}^{kl}_n \otimes \nabla \varphi_n.
$$
Plugging these Jacobian expressions into the double contraction we wish to compute:
\begin{align*}
\tau^{kl} : (\nabla p^{kl} \nabla v)
    &= \sum_{n,m} \tau^{kl} : \left[(\vec{p}^{kl}_n \otimes \nabla \varphi_n) (\delta \q_m \otimes \nabla \lambda_m)\right] \\
    &= \sum_{m} \delta \q_m \cdot
            \left(\sum_n \left[\nabla \lambda_m \cdot (\tau^{kl} \vec{p}^{kl}_n) \right] \nabla \varphi_n\right).
\end{align*}
Finally, we make these substitutions in $dJ_d[v]$ to express the
differential form as an inner product with the vertex node perturbations (here
summation over vertices $m$ and FEM nodes $n$ is implied).
\begin{equation*}
\boxed{
\begin{aligned}
dJ_d[\lambda_m \delta \q_m]
&=
\left(\vint{\shape}{
     \left[j -\strain(p^{kl}) : \sigma^{kl} \right] \nabla \lambda_m 
     + \left[
         \nabla \lambda_m \cdot \bigg(
                \sigma^{kl} \vec{\p}^{kl}_n +
                (\strain(p^{kl}) : C - \tau^{kl}) \w^{kl}_n
        \bigg) \right] \nabla \varphi_n
}\right) \cdot \delta \q_m
\\
&+ \left(\vint{\shape}{\gamma}\right) :: dC^H[v].
\end{aligned}
}
\end{equation*}
The transformation is completed once we substitute the formula derived in
\texttt{DiscreteHomogenizedTensorGradient.tex}:
$$
dC^H_{ijkl}[\lambda_m \delta \q_m] = 
    \left(
        \frac{1}{|Y|} \vint{\omega}{\left(\sigma^{ij} : C^{-1} : \sigma^{kl}\right) \nabla \lambda_m
           - \left[\nabla \lambda_m \cdot \left( \sigma^{kl} \vec{\w}^{ij}_n
                                         + \sigma^{ij} \vec{\w}^{kl}_n
                             \right) \right] \nabla \varphi_n}
    \right) \cdot \delta \q_m,
$$
where again summation over $m$ and $n$ is implied.

\bibliographystyle{plain}
\bibliography{MicrostructureWorstCase}

\end{document}
