\documentclass[10pt]{article}

\usepackage[latin1]{inputenc}
\usepackage{amsmath, amssymb, amsfonts, amsthm}
\usepackage{upgreek}
\usepackage{amsthm}
\usepackage{fullpage}
\usepackage{graphicx}
\usepackage{cancel}
\usepackage{subfigure}
\usepackage{mathrsfs}
\usepackage{outlines}
\usepackage[font={sf,it}, labelfont={sf,bf}, labelsep=space, belowskip=5pt]{caption}
\usepackage{hyperref}
% \usepackage{minted}
\usepackage{titling}
\usepackage{xifthen}
\usepackage{empheq}

\usepackage{fancyhdr}
\usepackage[title]{appendix}
\usepackage{float}
\usepackage[inline]{enumitem}

\DeclareMathOperator{\sgn}{sgn}
\DeclareMathOperator{\tr}{tr}
\DeclareMathOperator{\diag}{diag}
\renewcommand{\Re}{\operatorname{Re}} \renewcommand{\Im}{\operatorname{Im}}
\allowdisplaybreaks

\pagestyle{fancy}
\headheight 24pt
\headsep    12pt
\lhead{Discrete Gradient of the Homogenized Elasticity Tensor}
\rhead{\today}
\fancyfoot[C]{} % hide the default page number at the bottom
\lfoot{}
\rfoot{\thepage}
\renewcommand{\headrulewidth}{0.4pt}
\renewcommand\footrulewidth{0.4pt}
\providecommand{\sym}[1]{\operatorname{sym}\!\left(#1\right)}
\providecommand{\abs}[1]{\lvert#1\rvert}
\providecommand{\norm}[1]{\lVert#1\rVert}
\providecommand{\dx}{\, \mathrm{d}x}
% \providecommand{\vint}[2]{\int_{#1} \! #2 \, \mathrm{d}x}
% \providecommand{\sint}[2]{\int_{\partial #1} \! #2 \, \mathrm{d}A}
\renewcommand{\div}{\nabla \cdot}
\providecommand{\grad}{\nabla}
% We're talking about microstructure shape \omega
\providecommand{\shape}{\omega}
\providecommand{\mesh}{\mathcal{M}}
\providecommand{\boundary}{\partial \shape}
\providecommand{\bdry}{\boundary}
\providecommand{\vint}[3][x]{\int_{#2} \! #3 \, \mathrm{d}#1}
\providecommand{\sint}[3][x]{\int_{#2} \! #3 \, \mathrm{d}A(#1)}
\providecommand{\pder}[2]{\frac{\partial #1}{\partial #2}}
\providecommand{\tder}[2]{\frac{\mathrm{d} #1}{\mathrm{d} #2}}
\providecommand{\evalat}[2]{\left.#1\right|_{#2}}
\providecommand{\MDer}[1]{D\!\left[#1\right]}
% \renewcommand{\vec}[1]{{\bf #1}}

\usepackage{prettyref}
\newrefformat{sec}{Section~\ref{#1}}
\newrefformat{tbl}{Table~\ref{#1}}
\newrefformat{fig}{Figure~\ref{#1}}
\newrefformat{chp}{Chapter~\ref{#1}}
\newrefformat{eqn}{\eqref{#1}}
\newrefformat{set}{\eqref{#1}}
\newrefformat{alg}{Algorithm~\ref{#1}}
\newrefformat{apx}{Appendix~\ref{#1}}
\newcommand\pr[1]{\prettyref{#1}}

\def\normal{\hat{n}}
\def\n{{\bf n}}
\def\x{{\bf x}}
\def\X{{\bf X}}
\def\y{{\bf y}}
\def\z{{\bf z}}
\def\u{{\bf u}}
\def\w{{\bf w}}
\def\p{{\bf p}}
\def\v{{\bf v}}
\def\e{{\bf e}}
\def\ue{{\bf u}^\e}
\def\strain{\varepsilon}
\def\stress{\sigma}
\def\d{\, \mathrm{d}}
\def\R{\, \mathbb{R}}
\def\L{\mathcal{L}}
\def\flat{\mathscr{F}}

\DeclareMathOperator*{\argmin}{argmin}
\newcommand{\defeq}{\vcentcolon=}
\newtheorem{lemma}{Lemma}

\makeatletter
\usepackage{mathtools}
\newcases{mycases}{\quad}{%
  \hfil$\m@th\displaystyle{##}$}{$\m@th\displaystyle{##}$\hfil}{\lbrace}{.}
\makeatother

\setlength{\droptitle}{-50pt}
\title{Discrete Gradient of the Homogenized Elasticity Tensor}
\author{Julian Panetta}

% BEGIN DOCUMENT
\begin{document}
\maketitle

We compute the exact ``gradient'' of the homogenized elasticity tensor with
respect to mesh vertex positions. This differs from the continuous shape
derivative,
$$
dC^H_{ijkl}[\v] = \frac{1}{|Y|} \sint{\omega}{(\strain(\w^{ij}) + e^{ij}) : C : (\strain(\w^{kl}) + e^{kl}) (\v \cdot \n)},
$$
in that the internal velocity {\em will} contribute slightly and we can recover
more than just the ``normal'' component of the gradient. We expect the discrete
version to be especially important when the boundary is non-smooth and vertex normals
become harder to define.

\section{Homogenized Elasticity Tensor}
Recall that the homogenized elasticity tensor can be written in energy form as:
$$
C_{ijkl}^H(\omega) = \frac{1}{|Y|} \vint{\omega}{\left[\strain(\w^{ij}) + e^{ij}\right] : C : \left[\strain(\w^{kl}) + e^{kl}\right]},
$$
where $\w^{ij}$ is the fluctuation displacement corresponding to the $ij^\text{th}$ constant strain
basis tensor, $e^{ij}$, determining the cell problem load. That is, $\w^{ij}$ solves the cell problem:
\begin{gather*}
      -\nabla \cdot (C^\text{base} : [\strain({\bf w}^{ij}) + e^{ij}]) = 0 \text{ in } \omega, \\
    [C^\text{base} : \strain({\bf w}^{ij})]\n  =  - [C^\text{base} : e^{ij}]\n \text{ on } \partial \omega \backslash \partial Y, \\
    {\bf w}^{ij}({\y})\ Y\text{-periodic}, \\
    \int_\omega \! {\bf w}^{ij}({\y})  \, \mathrm{d} {\y} =  {\bf 0}.
\end{gather*}

\section{Differential Form}
Now we wish to compute how $C^H$ changes as the mesh vertices are perturbed.
This is most conveniently encoded as the tensor-valued differential one-form
$dC^H[\v]$ evaluated at the current microstructure $\omega$:
\begin{align}
    \label{eqn:continuous_diff}
    dC^H_{ijkl}[\v] \defeq &\evalat{\tder{}{\epsilon}}{\epsilon = 0} C^H_{ijkl}(\omega + \epsilon \v) \notag \\
    = &\frac{1}{|Y|} \vint{\omega}{\left(\left[\strain(\w^{ij}) + e^{ij}\right] : C : \left[\strain(\w^{kl}) + e^{kl}\right]\right) \div \v + \MDer{(\strain(\w^{ij}) + e^{ij}) : C : (\strain(\w^{kl}) + e^{kl})}},
\end{align}
where $\MDer{\cdot}$ denotes the material/Lagrangian derivative ($\epsilon$ is the
``time'' parameter).  The first term computes the change in $C^H$ due to the dilation at each
(material) point of integration domain ($\div \v$ computes the dilation), and the second
term computes the effect of the changing integrand at each material point.

\section{Discrete Version}
In the discrete setting, $\v$ is a piecewise linear velocity field
(even when using higher order, but straight-edged, finite elements):
$$
\v = \lambda^q \delta \p^q,
$$
where $\lambda^q$ is the $q^\text{th}$ mesh vertex's linear shape function and
$\delta \p^q$ is the vertex's offset.

Formula \pr{eqn:continuous_diff} also holds in this
setting: the integral becomes a sum of individual per-element integrals and $\div
\v$ is a piecewise constant scalar that measures the change in each element's
volume. The formula can be written more explicitly using the vector-valued
finite element shape functions, $\vec{\phi}$.
First we define the ``mutual energy'' density:
$$
M_{(ij,kl)} = \left[\strain\left(\w^{ij}_{(n,c)} \vec{\phi}^{(n,c)}\right) + e^{ij}\right] : C : \left[\strain\left(\w^{kl}_{(m,d)} \vec{\phi}^{(n,d)}\right) + e^{kl}\right],
$$
where $\vec{\phi}^{(n,c)} = \phi^n \e^c$ is the vector field with all components zero except
the $c^\text{th}$, which equals the scalar shape function $\phi^n$
corresponding to the $n^\text{th}$ node. Then we can write the differential more compactly:
$$
dC^H_{ijkl}[\v] = \frac{1}{|Y|} \vint{\omega}{M_{(ij,kl)} \div \v + \MDer{M_{(ij,kl)}}}.
$$
We can write $\div \v$ in terms of the vertex offsets:
$$
\boxed{
\div \v = \div \left(\lambda^q \delta \p^q\right) = \grad \lambda^q \cdot \delta \p^q
}
\quad \quad (\delta \p^q \text{ is spatially constant}),
$$
which indeed is a piecewise constant scalar that equals the change in volume of each element.
Now we simplify $\MDer{M_{(ij,kl)}}$:
\begin{align*}
\MDer{M_{(ij,kl)}}
&=
    \MDer{\w^{ij}_{(n,c)} \strain\left(\vec{\phi}^{(n,c)}\right)} : \sigma^{kl} +
    \sigma^{ij} : \MDer{\w^{kl}_{(m,d)} \strain\left(\vec{\phi}^{(m,d)}\right)}
    \\
&=
    \left(
        \MDer{\w^{ij}_{(n,c)}} \strain\left(\vec{\phi}^{(n,c)}\right) +
        \w^{ij}_{(n,c)} \MDer{\strain\left(\vec{\phi}^{(n,c)}\right)}
    \right) : \sigma^{kl} \\
    &\quad +\, \, \sigma^{ij} :
    \left(
        \MDer{\w^{kl}_{(m,d)}} \strain\left(\vec{\phi}^{(m,d)}\right) +
        \w^{kl}_{(m,d)} \MDer{\strain\left(\vec{\phi}^{(m,d)}\right)}
    \right),
\end{align*}
where $\sigma^{ij} \defeq C : \strain\left(\w^{ij}_{(n,c)}
\vec{\phi}^{(n,c)}\right) + e^{ij}$ is the microscopic stress corresponding to
probe $e^{ij}$. Here we used linearity of the strain and product rule for $\MDer{\cdot}$.
Notice that, since $\w^{kl}$ solves the $kl^\text{th}$ cell problem:
$$
\vint{\omega}{\strain\left(\vec{\phi}^{(n,c)}\right) : \sigma^{kl}} = 
\vint{\omega}{\strain\left(\vec{\phi}^{(n,c)}\right) : C : \left[\strain\left(\w^{kl}_{(m,d)} \vec{\phi}^{(m,d)}\right) + e^{kl}\right]} = 0 \quad \forall (n,c),
$$
thus all terms of $\MDer{M_{(ij,kl)}}$ involving
$\MDer{\w^{ij}_{(n,c)}}$ will integrate to zero in
\pr{eqn:continuous_diff} (these integrated terms can be written as linear
combinations of the integrals shown to be zero above). In other words, we do
not need to know the shape derivative of the fluctuation displacements in order
to compute the elasticity tensor's shape derivative (otherwise we would
need to solve an adjoint PDE to express the gradient in a computationally
tractable way). The remaining terms (those not integrating to zero) are:
$$
\underbrace{\sigma^{kl} : \w^{ij}_{(n,c)} \MDer{\strain\left(\vec{\phi}^{(n,c)}\right)}}_{(*)} +
    \sigma^{ij} : \w^{kl}_{(m,d)} \MDer{\strain\left(\vec{\phi}^{(m,d)}\right)}.
$$
We now simplify term $(*)$. Recall that $\strain(\vec{\phi}) = \sym{\grad \vec{\phi}}$. By linearity of $\MDer{\cdot}$,
$\MDer{\strain(\vec{\phi})} = \sym{\MDer{\grad \vec{\phi}}}$, and then symmetry of
$\sigma^{kl}$ we can actually drop the symmetrization and compute:
$$
(*) = \sigma^{kl} : \w^{ij}_{(n,c)} \MDer{\grad\vec{\phi}^{(n,c)}}
  = \sigma^{kl} : \w^{ij}_{(n,c)} \MDer{\grad \left(\phi^n \e^c\right)}
  = \sigma^{kl} : \w^{ij}_{(n,c)} \MDer{\e^c \otimes \grad \phi^n}
  = \sigma^{kl} : \left(\vec{\w}^{ij}_n \otimes \MDer{\grad \phi^n}\right).
$$
As shown in Appendix \ref{sec:DSFGrad}, we can write the material derivative of
shape function gradient in terms of the mesh vertex perturbations:
\begin{align*}
    (*) &= -\sigma^{kl} : \left(\vec{\w}^{ij}_n \otimes \left[\left(\grad \phi^n \cdot \delta \p^q\right) \grad \lambda^q \right]\right) \\
    &= -\left[\grad \lambda^q \cdot \sigma^{kl} \vec{\w}^{ij}_n \right] \grad \phi^n \cdot \delta \p^q.
\end{align*}

\section{Final Result}
We can finally combine the quantities above to express the homogenized elasticity
tensor's differential in terms of the mesh vertex perturbations:

Thus the homogenized elasticity tensor's differential can be written as:
\begin{align*}
    dC^H_{ijkl}[\lambda^q \delta \p^q]
    &= \frac{1}{|Y|} \vint{\omega}{M_{(ij,kl)} \grad \lambda^q \cdot \delta \p^q + \MDer{M_{(ij, kl)}}}, \\
    &= \frac{1}{|Y|} \vint{\omega}{M_{(ij,kl)} \grad \lambda^q \cdot \delta \p^q
            - \left(\grad \lambda^q \cdot \sigma^{kl} \vec{\w}^{ij}_n \right) \grad \phi^n(\x) \cdot \delta \p^q
            - \left(\grad \lambda^q \cdot \sigma^{ij} \vec{\w}^{kl}_n \right) \grad \phi^n(\x) \cdot \delta \p^q}, \\
    &=
\boxed{
    \left[
       \frac{1}{|Y|} \vint{\omega}{M_{(ij,kl)} \grad \lambda^q
           - \left[\grad \lambda^q \cdot \left( \sigma^{kl} \vec{\w}^{ij}_n
                                         + \sigma^{ij} \vec{\w}^{kl}_n
                             \right) \right] \grad \phi^n(\x)}
    \right] \cdot \delta \p^q.
}
\end{align*}
We could represent this scalar-valued one-form for tensor coefficient
$C^H_{ijkl}$ by storing the bracketed, vector-valued integral for each vertex
$q$ (as if it were a per-vertex vector field). Then applying the one-form to a
vertex perturbation field amounts to summing the dot products of the two
quantities stored at each vertex (one-form vector and perturbation vector).

Instead, we ``transpose'' this representation and store a tensor-valued one-form
(rather than one scalar-valued one-form for each tensor coefficient). This new
representation allows us to operate on the full object as a tensor instead of
being forced to process and manage each tensor component separately. 

Under this ``transposed'' representation, our data structure is analogous to a
per-vertex vector field, but instead of storing a scalar for each vertex
component, we store a tensor. That is, we store the full tensor
$\pder{C^H}{p^q_c}$ in $c^\text{th}$ component slot of vertex $q$ (the $ijkl^\text{th}$
entry of this tensor is given by the $c^\text{th}$ vector component of the bracketed integral).
Then, applying this one-form to a perturbation vector field is still like
computing an inner product: we compute a linear combination of all stored
tensors using as weights the corresponding components of the per-vertex
perturbation vectors.

\begin{appendices}

\section{Material Derivative of Shape Function Gradients}
\label{sec:DSFGrad}
To compute the material derivative, we need to establish the correspondences
between the perturbed elements  and the unperturbed
elements. Of course, this is determined by the piecewise linear velocity field
$\v$, which for a particular element $T$ is just the linear function $\v_T =
\lambda^s \delta \p^s$ where $s$ indexes $T$'s corner vertices. The map from
points on $T$ to perturbed element $\tilde{T^\epsilon}$ (at ``time''
$\epsilon$) is
$$
f^\epsilon(\x) = \x + \epsilon \lambda^s(\x) \delta \p^s,
$$
whose spatially constant Jacobian is:
$$
F^\epsilon = I + \epsilon \delta \p^s \otimes \grad \lambda^s.
$$
Since the Lagrange shape functions are expressed directly in terms of barycentric
coordinates, they satisfy the identity:
$$
\tilde\phi^\epsilon (f^\epsilon(\x)) = \phi(\x)
$$
(i.e. the shape functions simply advect with the geometry and their material
derivative is zero).
The gradient of the perturbed shape function $\tilde \phi^\epsilon$ at point
$f^\epsilon(\x)$ can then be written as:
$$
\left[\evalat{\grad_{\x^\epsilon}\tilde \phi^\epsilon}{f^\epsilon(\x)}\right]_a =
\evalat{\pder{\phi}{x_b}}{\x} \pder{x_b}{x^\epsilon_a} =
\left[\evalat{\grad \phi}{\x}\right]_b \left[(F^\epsilon)^{-1}\right]_{ba} =
\left[(F^\epsilon)^{-T} \evalat{\grad \phi}{\x}\right]_a.
$$
Thus, the material derivative is given by:
$$
\MDer{\grad \phi(\x)} \defeq
  \evalat{\tder{}{\epsilon}}{\epsilon = 0}\evalat{\grad_{\x^\epsilon}\tilde \phi^\epsilon}{f^\epsilon(\x)}
  = \left[\evalat{\tder{}{\epsilon}}{\epsilon = 0}(F^\epsilon)^{-T}\right] \grad \phi(\x)
$$
Using the identity $\tder{(A^\epsilon)^{-T}}{\epsilon} = - A^{-T} \tder{A^T}{\epsilon} A^{-T}$
and the facts that $\tder{F^\epsilon}{\epsilon} = \delta \p^s \otimes \grad
\lambda^s$ and $\evalat{F^\epsilon}{\epsilon = 0} = I$, we arrive at:
$$
\MDer{\grad \phi(\x)} = -\left(\grad \lambda^s \otimes \delta \p^s\right) \grad \phi(\x)
=
\boxed{
    -\left(\grad \phi(\x) \cdot \delta \p^s\right) \grad \lambda^s.
}
$$

\end{appendices}

\end{document}
